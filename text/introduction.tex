Porting of the applications becomes very current issue in the modern world of information technology. Every operating
system uses its own interface and technologies for application development. However, there are cross-platform tools that
allow porting of the applications without many modifications. One of these tools is the Java programming language.

Java language is developed by Oracle Corporation and it is distributed in several platforms. The most common platform is
Java Standard Edition (SE) which contains many of the basic Java libraries commonly used in standard desktop
applications. A~set of these libraries is called the Java SE Application Programming Interface (API).

Android is an~operating system for mobile devices developed by Google. It uses the Java programming language to
an~application development. Android runtime environment includes not only the libraries for development of graphical
user interface but also a~subset of Java SE API libraries.

OptaPlanner is an~open-source software developed by JBoss community designed for solving planning problems. It is
completely written in Java language and it is easily portable between desktop operating systems. However, Android API
does not contain all of the Java SE API libraries and therefore porting of the OptaPlanner tool causes problems with
dependencies. This thesis deals with these problems and shows an~implementation of a~simple Android application which
uses OptaPlanner tool to solve the Vehicle Routing Problem.

Along with this introduction, this thesis is divided into another seven chapters. Chapter~\ref{JavaChapter} presents the
Java programming language and its platforms. Chapter~\ref{AndroidChapter} describes the Android platform, its
architecture and the build process. Chapter~\ref{OptaPlannerChapter} deals with the OptaPlanner tool and shows how it
can be used. In Chapter~\ref{PortingChapter}, differences between Java SE API and Android API are described and the
possible solutions of the JavaBeans problem on the Android platform are suggested. One of the solutions is selected and
used for realization of the portation. Design and implementation of model Android Vehicle Routing Problem application
using OptaPlanner tool is presented in Chapter~\ref{ApplicationChapter}. Performed testings and future work is
described in Chapter~\ref{TestingChapter} and the last Chapter~\ref{ConclusionChapter} summarizes the entire work.
