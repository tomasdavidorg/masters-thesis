\section{OptaPlanner}

V této kapitole se zaměříme na planning engine Optaplanner. OptaPlanner je 
\todo{co je to optaplanner atd}


\subsection{Planning problem}

\subsection{Planner configuration}
Planner configuration se dá rozdělit na pět základních kroků, které jsou potřeba udělat před získáním řešení. Tyto kroky jsou:
\begin{enumerate}
\item \textbf{Namodelování plánovacího problému} -- vytvoření třídy, která ipmlementuje rozhraní Solution, tříd plánovacích entit a plánovacích proměnných
\item \textbf{Konfigurace Solveru} -- spočívá v nastavení pravidel a algoritmů řešeného problému
\item \textbf{Náhrání data setu problému} -- přirazení instancí plánovacích entit a proměnných
\item \textbf{Spuštění Solveru} -- zapnutí mechanismu pro řešení problému
\item \textbf{Získání nejlepšího řešení} -- vyvolání metody, která vrátí nejlepší získané řešení.
\end{enumerate}

\subsubsection{Namodelování plánovacího problému}
Při modelování plánovacího problému je dobré si ujasnit co je plánovací fakt a co entita. Problem fakt se během plánování nemění tedy jeho hodnoty zůstávají stále stejné. Oproti tomu je plánovací entita 
\paragraph{Problem fact}
\paragraph{Planning entity}
\paragraph{Planning variable}
\paragraph{Planning value and planning value ranges}
\paragraph{Planning problem and planning solution}

\subsubsection{Konfigurace Solveru}

\subsection{Score calculation}

