This chapter shows how to add Optaplanner dependency to Android project. It describes differences between Java SE API and Android API. Finally, it looks for a solution of JavaBeans problem. 

\section{Addition of Optaplanner dependency to project}
Application on Android platform use in recent versions build system gradle which can use maven dependencies. One of the distribution that the core OptaPlanner is distributed is the form of Maven repository. As we can see on Listing \ref{mvndep}, simple way how to connect to Optaplannet to Android application is adding maven dependence to build.gradle file to section depedencies.
%Aplikace na platformě android využívají v posledních verzích buildovací systém gradle, který umí využívat závislosti typu Maven. Jednou z distribucí, kterou je jádro OptaPlanner sitribuováno je ve formě  Maven repozitáře. Jak je možno vidět v Listings \ref{mvndep}, jednoduchý způsob jak připojit Optaplannet do anroid aplikace je přidáním maven závislosti do build.gradle souboru do sekce depedencies. 
\\
\begin{lstlisting}[captionpos={b},caption={Maven Optaplanner dependency.},frame={lines},label={mvndep},basicstyle=\footnotesize]
dependencies {
    ...
    compile 'org.optaplanner:optaplanner-core:6.1.0.Final'
    ...
}
\end{lstlisting}

\section{Java SE API and Android API}\label{apis}
Android API is based on Apache Harmony Java SE, but as we can see in Table \ref{javaDiff} it is not complete Java SE API. A significant part of the package are missing or some are incomplete. In the case of a package for the graphical user interface such as java.swing and java.awt were replaced with android graphical user elements. GUI is often separated from the computational part of the application and therefore when porting applications this is not a major problem. The real problem occurs when applications will require class dependencies that are not in the android api
%Android API je postaveno na Apache Harmony Java SE, avšak jak je možno vidět na tabulce \ref{javaDiff} není kompletní Java SE API. Podstatná část balíčků chybí nebo jsou některé ne zcela kompletní. V případě balíčku uživatelského rozhraní jako je java.awt a java swing byly nahrazeny androidími grafickými uživatelskými prvky. GUI  bývá často odděleno od výpočetní části aplikace, a proto při portování aplikací toto není hlavní problém. Skutečný problém nastává when aplikace bude vyžadovat třídní závislosti, které nejsou v android api.

\begin {table}[h!]
\begin{tabular}{|l|l|}
\hline
{\bf Java 6 SE Package} & {\bf Included in Android API} \\
\hline \hline
java.applet             & No -- missing completely    \\
java.awt                & Yes -- incomplete            \\
java.beans              & Yes -- incomplete            \\
java.io	                & Yes -- complete            \\
java.lang	            & Yes -- incomplete            \\
java.math	            & Yes -- complete            \\
java.net	            & Yes -- complete             \\
java.nio	            & Yes -- complete             \\
java.rmi	            & No  -- missing completely    \\
java.security	        & Yes -- incomplete            \\
java.sql	            & Yes -- complete            \\
java.text	            & Yes -- complete              \\
java.util	            & Yes -- incomplete            \\
javax.accessibility     & No -- missing completely    \\
javax.activation	    & No -- missing completely    \\
javax.activity	        & No -- missing completely    \\
javax.annotation	    & No -- missing completely    \\
javax.crypto	        & Yes -- complete              \\
javax.imageio	        & No -- missing completely    \\
javax.jws	            & No -- missing completely    \\
javax.lang	            & No -- missing completely    \\
javax.management        & No -- missing completely    \\
javax.naming	        & No -- missing completely    \\
javax.net	            & Yes -- complete              \\
javax.print		        & No -- missing completely    \\
javax.rmi		        & No -- missing completely    \\
javax.script	        & No -- missing completely    \\
javax.security          & Yes -- incomplete            \\
javax.sound             & No -- missing completely    \\
javax.sql	            & Yes -- incomplete javax.sql  \\
javax.swing	            & No -- missing completely    \\
javax.tools	            & No -- missing completely    \\
javax.transaction	    & No -- missing completely    \\
javax.xml	            & Yes -- incomplete            \\
org.ietf.jgss	        & No -- missing completely    \\
org.omg                 & No -- missing completely    \\
org.w3c.dom             & Yes -- incomplete            \\
org.xml.sax	            & Yes -- complete              \\
\hline
\end{tabular}
\centering
\caption{Java 6 SE packages in Android API}
\label{javaDiff}
\end{table}

\section{JavaBeans}
OptaPlanner is completely written in the Java language and one of its dependencies are classes from java.beans package. As can be seen in the table, this package is incomplete. When you run a simple OptaPlanner project on Android platform, ClassNotFoundException is thrown, precisely because the necessary classes in this package are not found. In this section we will see how this what are the possible solutions to this problem.
%OptaPlanner je kompletně napsán v programovacím jazyce Java a jednou z jeho závislostí jsou třídy z balíčeku java.beans. Jak je možno vidět v tabulce tento balíček je nekompletní. Při spuštění jednoduchého OptaPlanner projektu vznikne vyjímka ClassNotFoundException, právě z důvodu, že potřebné třídy se v tomto balíčku nenacházejí.   V této sekci si ukážeme jak tento jaké jsou možnosti řešení tohoto problému.

\subsection{Repacking of java.beans redistribution to java namespace}
The first of the ways how to replace the missing package java.beans is use of java.beans redistribution. These libraries are specially designed for the Android platform to support JavaBeans or other missing packages. If we do not want to change optaplanner code we will need to repackage these libraries to java namespace. In the following paragraphs we will introduce two redistribution and a tool for changing jar files -- Jar Jar Links.
%Prvním ze způsobů jak nahradit chybějící balík java.beans je využití redistribuce JavaBeans. Jedná se knihovny speciálně určené pro Android platformu pro podporu JavaBeans či dalších knihoven. Tyto knihovny je potřeba přebalit pomocí nástroje JarJar Links do namespace java.beans a výsledný jar soubor přiložit k android projektu. 

\subsubsection{OpenBeans}
OpenBeans is redistribution of java.beans package from the Apache Harmony project, which was created precisely because missing JavaBean on Android platfor. Used namespace of this redistribution is com.googlecode.openbeans namespace. It is an open source project and it is distributed as jar package that can be included into android project. 
%OpenBeans jsou redistribuce java.beans balíčku z Apache Harmony projektu. Namespace se ale od java.beans balíšku líší, v openbeans se používá com.googlecode.openbeans namespace. Vznikly v prvé řadě právě díky neexistenci java.beans na platformě Android. Jedná se o opensource projekt a je distribuován jako jar balíček, který je možné přidat do svého projektu. 

\subsubsection{Mad Robot}
A similar project as OpenBeans called Mad Robot. Like OpenBeans contains redistribution of java.beans package in com.madrobot.beans namespace. It also contains some additional packeges such as for database, graphic, geometry manipulation, etc. This project is distributed as Maven dependencies.
%Podobný projekt jako je OpenBeans se nazývá Mad Robot. Stejně jako OpenBeans obsahuje redistribuci balíčku java.beans pod namespace com.madrobot.beans a navíc přidává i některé další balíčky např pro práci s databází, s grafikou, geometrií atd. Tento projekt je distribuován ve formě Maven závislostí.

\subsubsection{Jar Jar Links}
Jar Jar Links is a utility for repackaging Java libraries. Enables repack java classes from one namespace to another. It is necessary to include rules file, which describe way how jar file should be repacked. Example of rule we can see on Listing \ref{rule}. Finally we just run the command with three parameters: rule file, input jar and output jar (Listing \ref{command}). 
%Jar Jar Links je nástroj pro přebalování Java knihoven. Umožňuje za přebalit java třídy z jednoho namespace do druhého. Proces probíhá pomocí pomocného souboru, kde se určí pravidla a způsob přebalení. Nakonec se pomocí příkazu spustí proces.
\\
\begin{lstlisting}[captionpos={b},caption={Jar Jar Links rule for repacking OpenBeans to java.beans namespace},frame={lines},label={rule},basicstyle=\footnotesize]
rule com.googlecode.openbeans.** java.beans.@1
\end{lstlisting}

\begin{lstlisting}[captionpos={b},caption={Command for repacking jar.},frame={lines},label={command},basicstyle=\footnotesize]
java -jar jarjar.jar process rule.txt openbeans.jar javabeans.jar
\end{lstlisting}

\subsection{Use the OpenJDK distribution source code}
This solution is based on the use of available source code of OpenJDK Java SE. So you can get the necessary libraries and add them directly to your project. The advantage of this solution is that we wil see the dependency failure in translation, not when running the application. This makes possible to choose source code we need. However, this adjustment is not trivial. Either it needs to use a special tool that removes unused dependencies or we must do it manually.
%Toto řešení zakládá na využití dostupných zdrojových kódu Java SE. Díky tomu je možné získat potřebné knihovny a přidat je přimo do svého projektu. Výhoda tohoto řešení je že se chyby závislostí ukáží už při překladu a ne až když beží aplikace. Díky tomu je možné si ze zdrojových kódu vybrat to potřebné. Tato upráva však není triviální a buď je potřeba použít speciální nástroje které odstraní nepoužívané závislosti nebo postup udělat manuálně. 

\subsection{Use of pruned rt.jar from OpenJDK distribution}
The last option without interfering with the source code is to use the package rt.jar which is part of the Java SE libraries. This package contains JavaBeans compiled classes and other parts of Java SE. Due to its size it is not well suited for android applications and also includes a libraries that are contained the Android API and thus causing collisions. Therefore, it needs to be pruned. The advantage of this pruning is that we do not have to worry about dependencies that are not needed for Optaplanner tool, because these files are not again compiled. On the other hand, it may happen that during runtime to hit the required dependencies and application throws an exception and the application crashes.
%Poslední možnoistí bez zásahu do zdrojového kódu je použití balíku rt.jar který je součást knihoven Java SE. Tento balík obsahuje zkompilované třídy JavaBeans a další součásti Java SE. Díky své velikosti se však příliš nehodí pro android aplikace a navíc obsahuje i knihovny které Android API obsahuje a proto by docházelo ke kolizím. Proto je nutné jej prořezat. Výhodou tohoto přořezání je že se nemusíme starat o závislosti, které nejsou potřeba pro Optaplanner nástroj, protože tyto soubory již neprocházejí java kompilátorem. Na druhou stranu se může stát že během běhu aplikace se narazí na potřebnou závislost a aplikace vyhodí vyjímku a aplikace spadne.

\subsection{Use of OpenBeans in OptaPlanner project}
The first option in which it is necessary to intervene in the Optaplanner source code is to replace dependencies java.beans for com.googlecode.openbeans by rewriting all java.beans imports and and by addition if openBeans.jar archive to Optaplanner project. All dependencies then will be redirected to OpenBeans. The disadvantage of this solution is the interference to OptaPlanner source code.  In terms of android application developers, we need to create a new fork of OptaPlanner and modify it and when new version of Optaplanner is released, we should merge changes. Maintenance is then considerably complicated.
%První možností při které je nutné zasahovat do zdrojového kódu je nahrazení závislostí java.beans za com.googlecode.openbeans. Tímto přepsáním importů a připojením balíčku openbeans.jar do jde k přeměrování na třídy open beans. Névyhodou tohoto řešení je právě zásah do zdrojových kódu optaplanneru. Z hlediska vývojáře android aplikace je potřeba vytvořit nový "fork" optaplanneru a ten upravit. A při nové verzi optaplanneru závést změny. Údržba je pak značně komplikovaná.

\subsection{Removind and replacing JavaBeans from OptaPlanner}
Last chance to solve the JavaBeans problem is his removal from the source code and its replacing by another technology. The disadvantages of this solution are primarily that this would be the biggest intervention to optaplanner code of the offered solutions.
%Poslední možnost vyřešení problému JavaBeans jeho odstranění ze zdrojového kódu a nahrazení jinou technologií. Nevýhody tohoto řešení zpočívají především v zásahu do zdrojových kódů. V podstatě by se jednalo o největší zásah z nábízených řešení. 

\subsection{Core library flag}
During the translation of android application that contains a class from namespace java. * or javax. *, translation crashes, which is highlighted by message that tell we use classes from java core namespace. This can be avoided by using the flag --core-library in the tool dx, which is located in android-sdk tools build folder. Adding a flag on the last line allow translation application. In Listing \ref{lastline} is possible to see how this line should look like.
%Při překladu android aplikace, která obsahuje třídy z namespace java.* nebo javax.* dojde k chybě, která upozorňuje na používání tříd z java core namespace. Tomuto se dá předejít použitím flagu --core-library v nástroji dx, který je umístěn v android-sdk ve složce build tools. Přidáním flagu na poslední řádek povolí překlad aplikace. V listungu je možné vidět jak tento řádek má vypadat.\todo{přidat zmínku k ostatním}
\\
\begin{lstlisting}[captionpos={b},caption={Spanning tree broadcast algorithm.},frame={lines},label={lastline},basicstyle=\footnotesize]
exec java $javaOpts -jar "$jarpath" --core-library "$@"
\end{lstlisting}

\subsection{Summary of approaches}
In Table \ref{advDis} we can see what are the advantages and disadvantages of each proposals. Furthermore, there are mentioned licenses, which should be respected when we choose specific solotion.
%V následující tabulce je možné vidět jaké jsou výhody a nevýhody jednotlivých řečení. Dále jsou přidány licence, které je potřeba respektovat při použití:
\begin {table}[h!]
\begin{tabular}{|p{2.5cm}|p{2cm}|p{2.4cm}|p{2.1cm}|p{5cm}|}
\hline
{\bf Approach} & {\bf Licence} & {\bf Optaplanner modification} & {\bf Complexity} & {\bf Notes} \\
\hline \hline
    Repacking OpenBeans redistribution to java namespace & Apache License 2.0 & No & Easy & 
    {\bf Advantages:} standalone jar file, no problems with dependencies \\
    \hline
    Repacking of Mad Robot redistribution to java namespace & LGPL 2.1 & No & Easy &
    {\bf Advantages:} same as in previous case \\
\hline
    Use the OpenJDK distribution source code &  GPL 2.0 & No & Hard &
    {\bf Advantages:} dependency failure occurs in translation, source code control

    {\bf Disadvantages:} difficult adjustment which can cause problems with dependencies \\
\hline
    Use of pruned rt.jar from OpenJDK distribution & GPL 2.0  & No & Hard &
    {\bf Advantages:} standalone jar file, 

    {\bf Disadvantages:}
    difficult adjustment which can cause problems with dependencies, incosistent jar, prunning \\
\hline
    Use of OpenBeans in OptaPlanner project & Apache License 2.0 & Yes & Easy &
    {\bf Advantages:} easy adjustment

    {\bf Disadvantages:} need of modification of Optaplanner source code and the subsequent maintenance of OptaPlanner fork \\
\hline
    Removind and replacing JavaBeans from OptaPlanner & -- & Yes & Medium &
    {\bf Disadvantages:} same as in previous case\\
\hline
\end{tabular}
\centering
\caption{Advantages and disadvantages of solutions of JavaBeans problem}
\label{advDis}
\end{table}

