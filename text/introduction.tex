Porting of applications becomes very actual issue in the modern world of information technology. Every operating system uses own interface and technologies for application development. However, there are tools that are cross-platform allowing porting the applications without many modifications. One of~these tools is the Java programming language.

Java languge is developed by Oracle Corporation and it is distributed in several platforms. The most common platform is Java Standart Edition which contains many of the~basic Java libraries which are ordinarily used in standard desktop applications. A set of~these libraries is called the Java SE API.

Android is and operating system for mobile devices developed by Google. It uses the~Java programming language to application development. Android runtime environment includes not only the libraries for development of user interface of Android applications but~also subset of Java SE API libraries.

OptaPlanner is an open source software developed by JBoss community designed for~solving planning problems. It is completely writen in Java and its easy portable between desktop operating systems. However, Android API does not contain all of the Java SE API libraries and therefore porting of the OptaPlanner tool may cause problems with dependencies.

This term project is divided into six chapters. Chapter~\ref{JavaChapter} presents the Java programing language and its platforms. Chapter~\ref{AndroidChapter} describes the Android platform, its architecture and the build process. Chapter~\ref{OptaPlannerChapter} explains what the OptaPlanner tool is and shows its configuration. In the Chapter~\ref{PortingChapter}, differences between Java SE API and Android API are described and the possible solutions of the JavaBeans problem on the Android platform are suggested. The last Chapter~\ref{ConclusionChapter} summarizes the entire work.

