In this project, the Java programing language, its most common platform Java Standard Edition, the Android operating system and the OptaPlanner tool were presented. Java is cross-platform language which is used for application development for Android operating system. Android is largely used as primary system for mobile devices and OptaPlanner is tool for solving optimalization problems. OptaPlanner is completely written in Java and it was not ported on Android yet. Therefore, this work focus on the portation of this tool and deals with a problems associated with it.

Application Programming Interfaces libraries of Java SE and Android API were compared and it was found that they differ significantly. Out of 38 Java SE API packages, 20 packages are completely missing and 9 packages are incomplete in Android API. On closer inspection, it was revealed that one of the OptaPlanner tool dependecies is missing in Android API. This missing package is named \texttt{java.beans} and it covers JavaBeans technology.

For JavaBeans problem were suggested five solutions, namely: Repacking JavaBeans redistribution to Java core namespace, Use of the OpenJDK distribution source code, Use of pruned rt.jar from OpenJDK distribution, Use of OpenBeans in OptaPlanner project and Removing and replacing JavaBeans from OptaPlanner. Due to comparison of advantages and disadvantages of individual solutions, it was selected a solution which does not change source code of Optaplanner tool, has a license that allows easy application and generally is suitable for application on Android.

The proposed solution for further progress is repacking of JavaBeans redistribution to Java core namespace. Next procedure will be portation of OptaPlanner core to Android platform and creating of simple Android application which demonstrates vehicle routing problem on ported tool. 
