Libraries of Java SE Application Programming Interfaces (API) and Android API were compared and it was found that they
differ significantly. Out of 38 Java SE API packages, 20 packages are completely missing and 9 packages are incomplete
in Android API. On closer inspection, it was revealed that one of the OptaPlanner tool dependecies is missing in
Android API. This missing package is named \texttt{java.beans} and it covers JavaBeans technology.

For JavaBeans problem were suggested five solutions, namely: Repacking JavaBeans redistribution to Java core namespace,
Use of the OpenJDK distribution source code, Use of pruned rt.jar from OpenJDK distribution, Use of OpenBeans in
OptaPlanner project and Removing and replacing JavaBeans from OptaPlanner. Due to comparison of advantages and
disadvantages of individual solutions, it was selected a solution which does not change source code of Optaplanner tool,
has a license that allows easy application and generally it is suitable for application on Android.

Chosen solution for further progress consists of repacking of OpenBeans redistribution to new JAR file which classes
are placed Java core namespace. Such file can be included in Android project and it supplement missing libraries.
However, this faces with problem which does not allow to use classes from Java core namespace on Android. The problem
was resolved by using of Core library flag. The entire process of the portation was designed, implemented and
automated by Gradle language -- the default build tool for Android projects.

According to the written requirements, sample Android Vehicle Routing Problem application which use the ported
OptaPlanner was designed and implemented. The application can use one of the three algorithms and set time limit to
solve attached problem examples. Actual problem is always displayed in graphic form on the screen of the application
together with the newest found solution.

The application is publicly available on Google Play Store on the internet and it was created presentation video which
was presented along with solution how to use OptaPlanner on Android to developers community on OptaPlanner website.

The challenge for the future work is portation of Drools tool. It is standalone project which can be used by OptaPlanner
as one of the option for calculation the score and in the future it may be interesting to have such tool on Android.
