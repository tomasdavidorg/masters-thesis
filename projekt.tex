%============================================================================
% tento soubor pouzijte jako zaklad
% (c) 2008 Michal Bidlo
% E-mail: bidlom AT fit vutbr cz
%============================================================================
% kodovaní: iso-8859-2 (zmena prikazem iconv, recode nebo cstocs)
%----------------------------------------------------------------------------
% zpracování: make, make pdf, make desky, make clean
% připomínky posílejte na e-mail: bidlom AT fit.vutbr.cz
% vim: set syntax=tex encoding=latin2:
%============================================================================
\documentclass[cover, english]{fitthesis} % odevzdani do wisu - odkazy, na ktere se da klikat
%\documentclass[cover,print]{fitthesis} % pro tisk - na odkazy se neda klikat
%\documentclass[english,print]{fitthesis} % pro tisk - na odkazy se neda klikat
%      \documentclass[english]{fitthesis}
% * Je-li prace psana v anglickem jazyce, je zapotrebi u tridy pouzit 
%   parametr english nasledovne:
%      \documentclass[english]{fitthesis}
% * Neprejete-li si vysazet na prvni strane dokumentu desky, zruste 
%   parametr cover

% zde zvolime kodovani, ve kterem je napsan text prace
% "latin2" pro iso8859-2 nebo "cp1250" pro windows-1250, "utf8" pro "utf-8"
%\usepackage{ucs}
\usepackage[utf8]{inputenc}
\usepackage[T1, IL2]{fontenc}
\usepackage{url}
\DeclareUrlCommand\url{\def\UrlLeft{<}\def\UrlRight{>} \urlstyle{tt}}

%zde muzeme vlozit vlastni balicky
\usepackage{todonotes}
\usepackage{listings}
\usepackage{graphicx}
\usepackage{multirow}

% =======================================================================
% balíček "hyperref" vytváří klikací odkazy v pdf, pokud tedy použijeme pdflatex
% problém je, že balíček hyperref musí být uveden jako poslední, takže nemůže
% být v šabloně
\ifWis
\ifx\pdfoutput\undefined % nejedeme pod pdflatexem
\else
  \usepackage{color}
  \usepackage[unicode,colorlinks,hyperindex,plainpages=false,pdftex]{hyperref}
  \definecolor{links}{rgb}{0.4,0.5,0}
  \definecolor{anchors}{rgb}{1,0,0}
  \def\AnchorColor{anchors}
  \def\LinkColor{links}
  \def\pdfBorderAttrs{/Border [0 0 0] }  % bez okrajů kolem odkazů
  \pdfcompresslevel=9
\fi
\fi

%Informace o praci/projektu
%---------------------------------------------------------------------------
\projectinfo{
  %Prace
  project=DP,            %typ prace BP/SP/DP/DR
  year=2015,             %rok
  date=\today,           %datum odevzdani
  %Nazev prace
  title.cs={Portace nástroje OptaPlanner na Android},  %nazev prace v cestine
  title.en={Port of OptaPlanner on Android}, %nazev prace v anglictine
  %Autor
  author={Tomáš David},   %jmeno prijmeni autora
  %author.title.p=Bc., %titul pred jmenem (nepovinne)
  %author.title.a=PhD, %titul za jmenem (nepovinne)
  %Ustav
  department=UITS, % doplnte prislusnou zkratku: UPSY/UIFS/UITS/UPGM
  %Skolitel
  supervisor=Zdeněk Letko, %jmeno prijmeni skolitele
  supervisor.title.p=Ing.,   %titul pred jmenem (nepovinne)
  supervisor.title.a={Ph.D.},    %titul za jmenem (nepovinne)
  %Klicova slova, abstrakty, prohlaseni a podekovani je mozne definovat 
  %bud pomoci nasledujicich parametru nebo pomoci vyhrazenych maker (viz dale)
  %===========================================================================
  %Klicova slova v ceskem jazyce
  keywords.cs={OptaPlanner, Android, Java, portace, Problém okružních jízd},
  %Klicova slova v anglickem jazyce
  keywords.en={OptaPlanner, Android, Java, portation, Vehicle Routing Problem},
  %Abstrakt v ceskem jazyce
  abstract.cs={Tato práce se zabývá portací nástroje OptaPlanner na operační systém Android. OptaPlanner je nástroj
  pro řešení plánovacích problémů a~je kompletně napsán v programovacím jazyce Java, který je také využíván pro vývoj
  aplikací operačního systém Android. Ten však neobsahuje všechny knihovny z Java Standard Edition Application
  Programming Interface a~při portaci nástroje OptaPlanner na Android tak dochází k problémům se závislostmi.
  Výsledkem této práce je návrh a~implementace řešení výše zmíněných problémů a~ukázková aplikace věnující se problému
  okružních jízd, který je řešen pomocí portovaného nástroje OptaPlanner.},
  %Abstrakt v anglickem jazyce
  abstract.en={This thesis deals with portation of the OptaPlanner tool to the Android operating system. The OptaPlanner is
  used for solving planning problems and it is completely written in the Java programming language which is also used
  for application development of the Android operating system. However, Android does not contain all of the
  Java Standard Edition Application Programming Interface libraries and porting of OptaPlanner to Android thus causes dependency problems.
  The result of the thesis is solution design and implementation of the problems mentioned above and model Android Vehicle Routing Problem
  application which uses ported OptaPlanner tool.},
  %Prohlaseni
  declaration={Prohlašuji, že jsem tuto diplomovou práci vypracoval samostatně pod vedením pana Ing. Zdeňka Letka,
  Ph.D. Další informace a~pomoc mi poskytl Geoffrey De Smet ze společnosti Red Hat. Uvedl jsem všechny literární prameny
  a~publikace, ze kterých jsem čerpal.},
  %Podekovani (nepovinne)
  acknowledgment={Na tomto místě bych rád poděkoval mému vedoucímu Ing. Zdeňku Letkovi, Ph.D. a~mému konzultantovi
  Geoffrey De Smetovi za~cenné rady, připomínky a~za čas, který mi věnovali. Díky patří také mé přítelkyni a~mým rodičům
  za jejich podporu a~pomoc při~studiu.} % nepovinne
}

%Abstrakt (cesky, anglicky)
%\abstract[cs]{Do tohoto odstavce bude zapsán výtah (abstrakt) práce v českém jazyce.}
%\abstract[en]{Do tohoto odstavce bude zapsán výtah (abstrakt) práce v anglickém jazyce.}

%Klicova slova (cesky, anglicky)
%\keywords[cs]{Sem budou zapsána jednotlivá klíčová slova v českém jazyce, oddělená čárkami.}
%\keywords[en]{Sem budou zapsána jednotlivá klíčová slova v anglickém jazyce, oddělená čárkami.}

%Prohlaseni
%\declaration{Prohlašuji, že jsem tuto bakalářskou práci vypracoval samostatně pod vedením pana X...
%Další informace mi poskytli...
%Uvedl jsem všechny literární prameny a publikace, ze kterých jsem čerpal.}

%Podekovani (nepovinne)
%\acknowledgment{V této sekci je možno uvést poděkování vedoucímu práce a těm, kteří poskytli odbornou pomoc
%(externí zadavatel, konzultant, apod.).}

\begin{document}
  % Vysazeni titulnich stran
  % ----------------------------------------------
  \maketitle
  % Obsah
  % ----------------------------------------------
  \tableofcontents
  
  % Seznam obrazku a tabulek (pokud prace obsahuje velke mnozstvi obrazku, tak se to hodi)
  % \listoffigures
  % \listoftables 

  % Text prace
  % ----------------------------------------------
  \chapter{Introduction}\label{IntroductionChapter}
\todo{introduction}

\cite{AndroidBook}
\cite{DroolsBook}
\cite{JavaBook}
\cite{AndroidDevBook}
\cite{AndroidProgBook}
\cite{OptaPlannerPages}
\cite{OptaPlannerDoc}
\cite{AndroidDev}
\cite{Java8Doc}
\cite{Java6Api}
\cite{AndroidArch}
\cite{JarJar}
\cite{OpenBeans}
\cite{MadRobot}



\chapter{Java}\label{JavaChapter}
This chapter introduces the Java programming language. It is primarily focused on description of its platforms and
specially on Java Standard Edition and its Application Programming Interface (API) which serves as the basis for Android
API described in Chapter~\ref{AndroidChapter}.

Section~\ref{JavaLangSection} presents the Java programming language, Java platforms are described in
Section~\ref{JavaPlatformsSection} and Section~\ref{JavaSESection} introduces Java Standard Edition and its main parts.

\section{Java language}\label{JavaLangSection}
Java~\cite{JavaBook, Java6Doc} is one of the most famous and most widely used computer programming languages in the
world. It is developed by Oracle Corporation and its application is widespread. Java is used for programming smart
cards, mobile and desktop applications, as well as large business and information systems. It is class based and object
oriented language which is managed by the Java Language Specification and together with the supporting runtime forms
programming environment.

The first public version of Java was released in 1996 and since this year, eight more versions was released. In 2010,
Java changed its owner from Sun Microsystems to Oracle Corporation. The latest version of Java (Java 8) was released in
2014.

\section{Java platforms}\label{JavaPlatformsSection}
Java is published in four platforms. Each platform provides tools for development and running programs written in Java
and consists of two main parts. The first part is Java Virtual Machine (JVM) which is connected to an~operating system
and thus Java programs can be executed. The second part is Java Application programming interface which provides many
public classes of standard Java libraries.

The following paragraphs briefly describe four platforms: Java Standard Edition (Java SE), Java Enterprise Edition
(Java EE), Java Micro Edition (Java ME) and Java Card.

\paragraph{Java Standard Edition}
Basic and the most famous platform which is designed for desktop and simple server application development. Currently,
the most recent version is Java SE 8.

\paragraph{Java Enterprise Edition}
Extension of Java SE that contains special libraries for developing and running enterprise software applications and
information systems. Java EE is based on Java SE 7 in the current version.

\paragraph{Java Micro Edition}
Subset of Java SE for application development for small devices such as microcontrollers, mobile phones, set-top boxes,
printers and other devices is called Java ME. Currently, the most recent version is Java ME 8.1.

\paragraph{Java Card}
This technology is designed for application development of smart cards or devices with limited memory and processing
capabilities. For example, it is used for SIM cards of mobile devices, plastic smart card for Automated teller machine
and similar devices. Last released version is Java Card 3.

\section{Java Standard Edition}\label{JavaSESection}
Java SE platform is distributed in two versions. The first version is Java Runtime Enviroment which is commonly used on
personal computers for running Java applications. The second version named Java Development Kit is used mostly for
application development. This platform has its own open-source implementation called OpenJDK~\cite{OpenJDK}.
Figure~\ref{JavaComponentsFigure} shows parts of Java SE Development Kit (JDK) distribution in comparison with Java SE
Runtime Environment (JRE) and in the following paragraphs, these components are briefly described.
\\
\begin{figure}[h!]
    \centering
    \includegraphics[scale=0.35]{fig/java_6_jdk.png}
    \caption{Components of the Java Standard Edition Development Kit 1.6~\cite{Java6Doc}.}
    \label{JavaComponentsFigure}
\end{figure}

\paragraph{Java SE Development Kit}
Java SE Development Kit is sometimes known as Software Development Kit (SDK). It is tool containing everything necessary
for developing Java application. The main part of JDK is JRE which is described below. Further, it contains the tools
to create and build applications (such as compiler, documentation generator, etc.), security, localization and other
tools. The last part of JDK is Java Language Specification which describes rules of this programming language.

\paragraph{Java SE Runtime Environment}
JRE is runtime environment for running Java applications. It consist from Java API, Java Virtual Machine and tools for
creating rich internet applications.

\paragraph{Java SE Application Programming Interface}
Java API is set of public classes of standard libraries. These libraries include packages for creating graphical user
interface, manipulation with databases, base language and utility libraries and many others.

\paragraph{Java Virtual Machine}
Java programs can not run without virtual machine. JVM is a~program that provides the runtime environment necessary for
executing of Java application. Figure~\ref{JavaLifecycleFigure} shows the lifecycle of Java program. It starts with
Java source code which is compiled to bytecode by \texttt{javac} compiler. Bytecode is an~instruction code and it is
stored in \texttt{.class} files. These files go through classloading mechanism to JVM and then are ready for execution
by the interpreter.
\\
\begin{figure}[h!]
    \centering
    \includegraphics[scale=0.3]{fig/java_program_lifecycle.png}
    \caption{The lifecycle of a~Java program~\cite{JavaBook}.}
    \label{JavaLifecycleFigure}
\end{figure}


\chapter{Android}\label{AndroidChapter}
Android is mobile operating system developed by Google. It's opensource system based on the Linux kernel mainly used in mobile devices such as smartphones, tablets and smart watches, but it can be found also in devices such as set-top boxes, media players and other electronics.
%Android je mobilní operační systém vyvíjený společností Google, který je založený na Linuxovém jádře. Je vyvíjen jako opensource a používá se především v mobilních zařízeních jako jsou chytré telefony, hodinky a tablety. Můžeme jej však také nalézt také v přístrojích jako jsou set-top boxy, multimediální přehrávače a v jiné elektronice.\\

\section{History}
Android, Inc. was founded in California USA in 2003. Google, Inc. bought Android two years later. In 2007, Google acquired several patents in the field of mobile devices. The same year on November 5, there is official presentation of an association of companies formed the Open Handset Alliance, which aims to create open standards in mobile devices. The first smartphone running Android released on October 22, 2008. Table \ref{androidHistory} presents a brief history of an operating system Android. 
%Počátky Androidu spadají do roku 2003, kdy byla v Kalifornii v USA založena společnost Android, Inc. O Dva roky později firmu odkupuje světoznámá společnost Google. V roce 2007 získává Google několik patentů v oblasti mobilních zařízení a 5. listopadu téhož roku dochází k oficiálnímu představení a vzniká sdružení firem Open Handset Alliance, která mé za cíl vytvoření otevřených standartů v oblasti mobilních zařízení. V tabulce odkaz je zobrazena stručná historie verzí operačního systému Android.

\begin {table}[h!]
    \begin{tabular}{|l|c|l|c|}
    \hline
    {\bf Release date}  & {\bf Version} & {\bf Codename}        & {\bf API level}   \\
    \hline \hline
    September 23, 2008  & 1.0 -- 1.1    & ---                   & 1 -- 2            \\
    \hline
    April 27, 2009      & 1.5           & Cupcake               & 3                 \\
    \hline
    September 15, 2009  & 1.6           & Donut                 & 4                 \\
    \hline
    October 26, 2009    & 2.0 -- 2.1    & Eclair                & 5 -- 7            \\
    \hline
    May 20, 2010        & 2.2 -- 2.2.3  & Froyo                 & 8                 \\
    \hline
    December 6, 2010    & 2.3 -- 2.3.7  & Gingerbread           & 9 -- 10           \\
    \hline
    February 22, 2011   & 3.0 -- 3.2    & Honeycomb             & 11 -- 13          \\
    \hline
    October 18, 2011    & 4.0 -- 4.0.4  & Ice Cream Sandwich    & 14 -- 15          \\
    \hline
    July 9, 2012        & 4.1 -- 4.3    & Jelly Bean            & 16 -- 18          \\
    \hline
    October 31, 2013    & 4.4 -- 4.4.4  & KitKat                & 19 -- 20          \\
    \hline
    November 12, 2014   & 5.0 -- 5.0.2  & Lollipop              & 21                \\
    \hline
    \end{tabular}
    \centering
    \caption{Android version history}
    \label{androidHistory}
\end{table}

\section{Architecture}
Architecture of Android system consists from six layers shown in Figure \ref{androidArchitecture}. We introduce these layers in the following section.
%Architektura systému android je složena z šesti vrstev:

\begin{figure}[h!]
    \centering
    \includegraphics[scale=0.5]{fig/android_architecture.jpg}
    \caption{Android architecture \cite{AndroidArch}}
    \label{androidArchitecture}
\end{figure}

\subsection{Linux kernel} %https://source.android.com/devices/
The lowest layer stands between hardware devices and other architectural layers. Android is based on special version of Linux kernel and several accessories such as memory management system, the Binder IPC driver and others. Since the beginning, Android was built on the Linux 2.6 kernel, the latest Android version runs on the kernel 3.4.
%Tato nejnižší vrstva je postavena mezi hardware zařízení a ostatní vrstvy architektury. Android je postavený na zvlaštní verzi Linuxového jádra a několika speciálními doplňky jako jsou správa systémové paměti, the Binder IPC driver, and other. Od počátku byl android postaven na linuxovém jádru 2.6, nejnověší android pak běží na jádru 3.4.  Při startu se jádro zavede do operační paměti a je mu předáno řízení.

\subsection{Hardware abstraction layer (HAL)} %https://source.android.com/devices/
Hardware abstraction layer is standart interface, which allows android system calls to drivers layer, while he does not care what is the implementation of in the lower layers drivers and hardware. For each piece of hardware should be a driver and matching HAL providing hardware options.
%HAL je standartní rozhraní které umožňuje systému android volat do vrstvy ovladačů, zatímco je mu jedno jaká je implementace v nižších vrstvách obladačů a hardwareru. Pro každý kus hardwaru by měl existovat ovladač a k němu odpovídající HAL poskytující možnosti hardwaru.

\subsection{Libraries}
Above the HAL is a layer of native libraries. These libraries are written in C or C ++. These libraries can be accessed through android sdk, but if direct access is required, it is possible to do this through the Native Development Kit (NDK). These libraries include:
%Nad HAL vrstvu se nachází vrstva nativních knihoven. Tyto knihovny jsou napsány v jazyce C nebo C++. K těmto knihonám se dá přistoupit přes android sdk pokud je však požadován přímý přístup je možné to provést přes NDK. Mezi tyto knihovny patří:

\begin{itemize} %http://www.android-app-market.com/android-architecture.html
\item Surface manager -- library for composing windows on the screen
%knihovna pro skládání oken na obrazovce
\item Media Framework -- provides various multimedia codecs for playing and recording video in various formats
%poskytuje různé multimediální kodeky pro nahrávání a přehrávání videa v různých formátech.
\item SQLite -- database engine for the use in data storage
%databázový engine pro použítí v oblasti uložení dat
\item WebKit -- a browser engine for displaying web content
%prohlížečový engine pro  zobrazování webového obsahu
\item Libc -- standard C library
%standartní knihovna jazyka C 
\item OpenGL ES -- library for support 2D and 3D graphics and rendering
%knihovna pro podporu 2D a 3D grafiky a renderování.
\item Audio Manager -- library for working with sounds of device
%knihovna pro práci se zvuky zařízení
\item FreeType -- library for bitmap and vector font rendering
%knihovna pro bitmapové a vektorové vykreslení písma
\item SSL -- library for the use of encryption protocol for secure Internet communications
%knihovna pro využití šifrovacího protokolu pro bezpečnou intenetovou komunikaci.
\item and others.
\end{itemize}

\subsection{Android runtime}
Android runtime layer is located next to native libraries. This layer consist from two parts: the Core of Libraies and Dalvik Virtual Machine (DVM). The core libraries can be further subdivided into two parts namely Java Libraries and Android library.
%Vedle vrstvy nativních knihoven se nachází se nachází Android Runtime vrstva, která se skládá ze dvou základních částí a to z Core Libraies a z Dalvik Virtual Machine (DVM). Core libraries jinak řečené Android API pak můžeme rozdělit ještě na dvě částí a to Java Knihovny a Android knihovny.

\subsubsection{Dalvik Virtual Machine (DVM)}
DVM is a virtual machine that is being developed since 2005 the system and it was included into the system due to the JVM is not licensed as open source. The second reason was the optimization for mobile devices. 
%DVM je virtuální stroj, který je vyvíjen od roku 2005 a byl do systému uzačleněn z důvodu, že JVM není licencován jako opensource a druhým důvodem byla optimalizace pro mobilní zařízení.

Each application runs on android devices within its own instance of DVM (not as a process in the Linux kernel).
%Každá aplikace beží na android zařízení v rámcí své vlastníá instance DVM, tedy nikoliv jako proces přímo v linuxovém jádře.

Running applications on a virtual machine has many advantages. First, it operates in a sandbox and thus can not interfere with the operating system or to other applications. Secondly, it makes the application platform-independent and tr can be run on any hardware. advantages include its DVM efficiency in memory usage and is thus better adapted for use on mobile devices.
%Spouštění aplikací na Virtuálních strojích přináší řadu výhod.  Jednak se jedná o to že pracují v izolovaném prostoru a tudíž nemohou zasáhnout do operačního systému či do jiných aplikací. Za druhé to činí z aplikace platformně nezávislou a tak může být zpuštěnan na jakémkoliv hardwaru. Mezi další výhody DVM patří jeho efektivnost ve využívání paměti a díky tomu je lépe přizpůsoben k použití na mobilních zařízeních.

The application code must be always transformed from a standard java file into Dalvik executables (.dex format) to be run into DVM. This provides dex tool, which performs this conversion. More information about this can be found in section \ref{buildProcess}.
%Proto aby mohla být aplikace spuštěna v DVM, kód aplikace musí být transformován ze standartních java souboru do dalvik executables (.dex format). K tomuto se používá dx nástroj ktérý tento převod vykonává.

\subsubsection{Java libraries}
Most Android applications are written using Java. These libraries are open source Java implementation of libraries based on Apache Harmony Project and is a subset of the Java SE platform. They do not contain for examole java.awt or java.swing libraries, which are replaced by their own components for creating Android applications user interface. A more detailed comparison can be found in the chapter \ref{apis}.
%Většina Android aplikací je napsána pomocí jazyka Java. Tyto Java knihovny jsou opensource implementaci knihoven založenou na Apache Harmony Projektu a jedná se o podmnožinu Java SE platformy. Neobsahují např. knihovny java.awt nebo java.swing, které jsou nahrazeny vlastními třídami pro tvorbu uživatelského rozhrani Android aplikací. Podrobnější srovnání je možné nalezt v kapitole TODO. 

\subsubsection{Android libraries}
These are specific libraries that provide all the functionality of android devices. Libraries are written in Java, and contains the following packages:
%Jedná se o specifické knihovny, které poskytují veškerou funkcionalitu android zařízení. Knihovny jsou napsány v jazyku Java a patří mezi ně tyto balíky:

\begin{itemize}%http://www.techotopia.com/index.php/An_Overview_of_the_Android_Architecture
\item android.app -- provides access to the application model and is the cornerstone of all applications
%poskytuje přístup k aplikačnímu modelu a je základním kamenem všech aplikací 
\item android.content -- classes for accessing and publishing data applications
%obsahují třídy pro přístup a publikování dat aplikací
\item android.database -- classes for data access and database manipulation 
%třídy pro přístup k datům a manipulaci s databazemi
\item android.graphics -- library for screen low-level 2D graphics drawing
%knihovny pro vykreslování na obrazovku
\item android.hardware -- provide access to hardware features such as cameras and sensors
%poskytují přístup k hardware funkcím jako jsou kamera a sensory
\item android.media -- library for handling with multimedia 
%knihovny pro práci s multimédii
\item android.text -- library for manipulate and rendering of strings
%knihovny pro praci s řetězci a ajejich zobrazením na display
\item android.util -- common tools such as data manipulation and time, conversions between numbers and strings, and more
%bežné nástroje jako manipulace s daty a časem převody mezi čísly a řetězci a další 
\item android.view -- basic building library for building a graphical user interface
%základní stavební blok pro budování grafického uživatelského rozhraní
\item android.webkit -- libraries for working with web content
%knihovny pro práci s webovým obsahem
\item and others.
\end{itemize}

\subsection{Application framework}
Application framework layer provides many high-level services to applications in the form of java libraries. For developers, this is the most important layer that allows access to the device. This layer consists of:
%Vrstva aplikačního frameworku poskytuje mnoho vysoko-úrovňových služeb aplikacím ve formě java knihoven. Pro vývojáře se jedná o nejdůležitější vrstvu,která umožňuje přístup ke službám daného zařízení. Tato vrstva je tvořena:
 
\begin{itemize} %http://developer.android.com/reference/
\item Activity manager -- controls all aspects of the application lifecycle
%ovládání životního cyklu aplikací jejich start průběh a konec
\item Windows manager --  windows management visibility and their arrangement
%pro správu viditelnosti oken a jejich uspořádání
\item Content Providers -- allows to work with the contents of other applications, provides mechanisms for security
%umožňuje pracovat s obsahem jiných aplikací, poskytuje mechanismy pro jejich zabezpečení 
\item View System -- View is a basic building block for user interface components. View system is a set of View that is used to build the application user interface.
%View je základní stavební blok pro komponenty uživatelského rozhraní. View systém je sada View která slouží k budování uživatelského rozhraní aplikace.
\item Notification manager -- allows user to inform through alerts and notifications
%Umožňuje uživatele informovat pomocí alertů a notifikací.
\item Package manager -- allows you to get various information about applications that are currently installed on device.
%Umožňuje získat různé informace o aplikacích které jsou aktuálně naisntalovány  zařízení.
\item Telephony manager -- provides access to telephone services of device
%Poskytuje přístup k telefoním službám zařízení. 
\item Resource manager -- allows access to resources such as color settings, layouts and strings
%Umožňuje přístup ke zdrojům jako jsou barevná nastavení, rozvržení a stringy.
\item Location manager -- provides access to the location services, these services allow you to periodically receive geographic coordinates of the device
%Poskytuje přístup k systémovým lokačním službám. Tyto služby umožňují v pravidelných intervalech získávat geografick souřadnice zařízení.
\item and others.
\end{itemize}

\subsection{Applications}
The last and highest layer consists of the application itself. These comprise both pre-installed applications and applications that have been added over time from the android store or an other way.
%Poslední a nejvyšší vrstva se skládá ze samotných aplíkací. Jendak se jedná o předinstalované aplikace a druhak se jedná o aplikace, které byly postupem času přidány z android obchodu nebo jinou cestou.

\section{Application structure}
In this section we will describe the anatomy of the application, various components that are used for creating android applications.
%V této sekci budou popsány hlavní části aplikace. Budou popsány jednotivé komponenty které se používají oro tvorbu android aplikací.

\subsection{Activities}%http://developer.android.com/guide/components/activities.html
Activities represent one single screen of user interface. Typically, after starting the application the main activity shows and from there we can run another activity or perform other operations. When you start a new activity, the previous one is stored in a LIFO stack. After pressing the back button is invoked again.
%Aktivity reprezentují jednu samotnou obrazovku živatelského rozhraní. Typicky po spuštění aplikace je zobrazena hlavní aktivita z které se pak mohou spouštět aktivity další, či provádět jiné operace. Když se spustí nová aktivita, ta předchozí je uschována v Last in first out zásobníku. Po stisknutí tlačítka zpět je znovu vyvolána.

\subsection{Services}
Services are components that run in the background performing long-term tasks. They do not provide a user interface. Services may be still active in the background while running other applications. Example of services can be download content from the Internet while another application is running.
%Služby jsou komponenty, které beží na pozadí provádějící dlohotrvající operace. Neposkytují uživatelské rozhraní. Služby mohou být stále aktivní i na pozadí zatímco beží jiná aplikace. Příkladem služby může být stahování obsahu z internetu zatímco je spuštěná jiná aplikace.

\subsection{Content providers} %http://developer.android.com/guide/components/fundamentals.html
Content providers store, load data and make it available for other applications. Through the content providers other applications may modify or manipulate specific data. An example might be a content provider that manages the contact information on the device.
%Content providers ukládají, načítají data a zpřístupňjí je pro všechny aplikace. Zkrz content providery mohou ostatní aplikace data modifikovat či s nimi manipulovat. Příkladem může být kontent provider který spravuje informace o kontaktech v zařízení.

\subsection{Intents}
The application area is composed of activities and messages between them, which we call Intents. Intent consist of activities to be done and the parameter that is attached to it. We can divide intents into explicit and implicit. Explicit Intent contain action and intention to be made and platform will select the application. Implicit Intent also contain action and object but depends on which application the user chooses.
%Aplikační prostor je složen s aktivit a zpráv mezi nimi, které nazýváme intenty. Ty se skládájá z činnosti která se má vykonat a parametru který je k ní připojen. Intenty můžeme rozdělit na explicitní a implicitní. Explicitní intenty obsahují akci a záměr které se má provést a výběr aplikace už zařídí platforma.  Implicitní intenty obsahují také akci a objekt ale závisí na uživateli jakou aplikaci zvolí.

\subsubsection{Broadcast receivers}
These are components which are used to listen notification from outside or from inside of the application. Reaction is formed by the type of notification: notification in the status bar, toast, or notification dialog box. An example might be a notification of incoming sms or low battery.
%Jsou to komponenty, které složí k nasloucháná oznámení z vnějšku popř. zevnitř aplikace. Podle typu oznámení následuje reakce: oznámeníve stavovém řádku, toast, či oznámení dialogovým oknem Příkladem může být oznámení o příchozí sms nebo nízkém stavu baterie.

\subsection{Application Resources} %http://developer.android.com/guide/topics/resources/available-resources.html
Android application consists not only from source code but also from the resources that are separated from the code. These resources include:
%Android aplikace se skládá nejen ze drojových kódů ale také z resources, které jsou od kódu odděleny. Mezi tyto resourcy patří:
\begin{enumerate}
\item Animation Resources -- defines predefined animation
%definují předem stanovené animace
\item Color State List Resource -- defines color change based on the View state
%definují barvy měnící se na základě stávu View
\item Drawable Resources -- defines different graphics -- bitmaps or xml files
%definují různé grafiky s bitmapami nebo xml soubory
\item Layout Resource -- defines the layout of application components
%definují rozložení component aplikace
\item Menu Resource -- defines content of aplication menus
%definují obsah menu aplikace
\item String Resources -- defines string, string arrays
%definují řetězce, pole řetěztců.
\item Style Resource -- defines the appearance and format of ui elements.
%definují vzhled a formát ui elementů.
\item and other resources.
\end{enumerate}

\subsection{Application Manifest} % http://developer.android.com/guide/topics/manifest/manifest-intro.html
Each application's root folder must have a file AndroidManifest.xml. This file contains information about the application with regard to Android. It should contain a unique package name for the application, the declaration of used components - activities, services. Then application permissions to the protected parts of the API (such as access to the camera, etc.). There also have to be declared a minimum level api, list of libraries with which the application is connected and other informations about application.
%Každá aplikace ve své rootovské složcě musí mít soubor AndroidManifest.xml. Tento soubor obsahuje informace o aplikaci s ohledem na systém Android. Měl by obsahovat unikátní jméno balíčku pro danou aplikacim deklaraci použitých komponent - aktivitm, služeb. Dále pak oprávnění aplikace ke chtáněným částem api (jako přístup k fotoaparátu apod). Také je potřeba deklarovat minimální úroveň api a seznam knihoven s kterými je aplikace propojena.

\section{Build System}%http://developer.android.com/sdk/installing/studio-build.html
Build System is the way how .apk package is produced from source code and resources. Apk file is the package file format used to distribute and install application software to Android. The entire process is automated by using Gradle scripts in the latest versions.
%Build System je způsob jakým se vyprodukuje ze zdrojového kódu a závislostí a reourců .apk balík. Celý tento proces je automatizován pomocí gradle skriptů ale je pro portování aplikací je dobré mu porozumět.

\subsection{Build Process}\label{buildProcess}
Figure \ref{buildProcess} shows way of creating .apk package. The individual steps are these:
%Ná obrázku je možné vidět jakým způsobem probíhá vytváření .apk balíčku. Jednotlivé kroky jsou pak tyto:
\begin{enumerate}
\item The Android Asset Packaging Tool compile resource files and AndroidManifest.xml and produce R.java file, so it is possible to refer to the resources.
%The Android Asset Packaging Tool zkompiluje resource soubory jako jsou manifest a xml soubory a vyprodukuje R.java takže je možné se na resourcy odkazovat.
\item Aidl coverts all .aidl interfacese to java interfaces.
%Aidl konvertuje všechny .aidl rozhraní do java rozhraní
\item Whole code is compilated by java compilator. Outputs are .class files. 
%Všechen kód je zkompilován java kompilátorem, výstupem jsou .class soubory.
\item Dex tool converts .class and third-party files into Dalvik byte code.
%Nástroj dex převede .class soubory do Dalvik byte kodu. Stejně tak soubory třetích stran
\item All uncompiled resource (eg. Images), compiled resource and .dex files are sent to apkbuilderu which is pack them to .apk file.
%Všechny nokompilované resourcy (např. obrázky), kompilované resourcy a .dex soubory jsou poslany apkbuilderu aby je zabalil do .apk souboru.
\item Then .apk file must be signed.
%Poté musím byt .apk podepsán.
\item Finally, if the application is being signed in release mode, zipalign tool align the .apk file and thereby decreases memory usage
%Nakonec pokud je podepsán v release módu musí být zipalign nástrojem vyrovnán a tím se sníží paměťová náročnost.
\end{enumerate}

\begin{figure}[h!]
    \centering
    \includegraphics[scale=0.7]{fig/build.png}
    \caption{Build process}
    \label{buildProcess}
\end{figure}





\chapter{OptaPlanner}\label{OptaPlannerChapter}
In this chapter, the planning engine Optaplanner is presented. OptaPlanner is an open source project developed by JBoss community since 2006 and it is software designed for solving planning problems. Optaplanner is a part of the Drools project and tt combines optimization algorithms with the core of rule engine (Drools Expert).

Section~\ref{planningProblem} shows what planning problem is and presents basic terminology. Introduction of basic phases of the OptaPlanner configuration is described in Section~\ref{plannerConf}. 

\section{Planning problem}\label{planningProblem}
In everyday life, at work or in another occasion, the people meet the problems for which they have limited resources (time, money, etc.). Also organizations need to face these problems at a larger scale. Planning mechanisims help them to save these resources and time. The planning problem is something what can be described and on what mechanisms can be applied.

In Optaplanner, the planning problem is represented by Java classes, XML configuration file and optional DRL rule files. Java classes together form the model of the planning problem. Configuration file is written in Extensible Markup Language and it serves to describe the configuration of Solver. Confiration contains declarated model classes, score configuration and used optimalization algorithm. DRL files are optional. They can be used for contains special rules calculation of a score otherwise the score could be calculated by a Java function.

Planning problems can be for example N-Queen problem, Vehicle routing problem, Course timetabling or Hospital bed planning. More examples can be found on \cite{}.

\paragraph{Solver}
A tool in OptaPlanner which solves optimalization problems is the Solver. It uses the problem model and calculates the score of possible initialized solutions. Score is a way how to compare two solution. Except of calculating the score, solver use optimalization algorithms to find the best score of planning problem. End of calculation can be caused for example by find the best solution or by reaching the time limit.

\paragraph{Solution}
An instance of the problems is called solution. There are two basic types of solution in OptaPlanner -- uninitialized and initialized solution. In contrast to second type, the first one does not have the calculated score yet.

\paragraph{Score}\label{score}
Score is a way how to compare two solutions of problem. Every solution has own score and solution with higher score is better. There is significant difference between best and optimal solution. Solver finds solution with the higest score from possible solutions -- the best solution but it is not always the optimal solution which is the best solution of current problem. There are several techniques for comparing scores:
\begin{itemize}
\item \textbf{Score constraint signum} -- based on constraints. Solver finds the highest score for positive constraint and try to reduce the negative value for negative constraints.
\item \textbf{Score constraint weight} -- technique where constraints may not have same weight and thus some of them can be more important then others. For example, the first condition is three times more important than the second condition.
\item \textbf{Score level} -- based on levels of score. Some scores are more important than others (Hard scores). Therefore, we compare them first and then we can decide by the less important scores (Soft scores).
\item \textbf{Pareto scoring} -- Score constraints cannot be weighted against each other therefore they are compared individually and the score with the most dominating score constraints wins.
\item \textbf{Combining score techniques} -- All the previous techniques can also be combined.
\end{itemize}

\paragraph{Optimization algorithms}\label{optimalAlg}
Every individual solution computation takes some time. OptaPlanner doesn't count only one solution but looking for the best solution and there may be a lot of solutions. The search space can grow to astronomical proportions, and the calculation time as well. More information about optimization algorithms can be found at \cite{OptaPlannerDoc}. Following list shows some of algorithms that can be used:
\begin{itemize}
\item \textbf{Exhaustive Search} -- Brute Force,  Branch And Bound
\item \textbf{Construction heuristics} --  First Fit, Weakest Fit,  Strongest Fit, \dots
\item \textbf{Metaheuristics}
\begin{itemize}
\item \textbf{Local Search} --  Hill Climbing, Tabu Search, Tabu Search, \dots
\item \textbf{Evolutionary Algorithms} -- Evolutionary Strategies, Genetic Algorithms
\end{itemize}
\end{itemize}

\paragraph{N-Queen problem}
One of planning problems is N-Queen problem. It is not very realistic case but it is ideal as an example. Problem is that it is necessary to place n queens to the chess field of size n. It is known that from the Chess game that queen can move vertically, horizontally and diagonally. The goal is that we try to achieve that none of the queens should threaten another.

\section{OptaPlanner configuration}\label{plannerConf}
In this section, OptaPlanner configuration is described. It can be divided into five basic steps that are required to get the best solution. The steps are the following:

\begin{enumerate}
\item \textbf{Modeling of planning problem} -- creation of a class that implements the Solution interface and definition of planning domain classes
\item \textbf{Solver configuration} -- settings of a score function, optimization algorithms and other parameters of a Solver
\item \textbf{Loading of problem data set} -- insertion of planning entities and variables instances into Solver
\item \textbf{Starting of Solver} -- start of mechanism for solving the problem and automatic calculation of scores
\item \textbf{Acquiring the best solution} -- method invocation, which returns the best obtained solution
\end{enumerate}

\subsection{Modeling of planning problem}
Modeling of planning problem consists of defining individual parts of the problem and creation of corresponding Java classes such as problem fact, planning entity and variable, planning solution and other classes.

\paragraph{Problem fact}
Problem fact is a class which contains getters returning its properties. This class does not contain special OptaPlanner code (it could be ordinary java class) and during planning it does not change. In case of N-Queen problem, rows and columns classes are problem facts.

\paragraph{Planning entity}
Planning entity is a class which change during planning. It has to be marked with @PlanningEntity annotation. Each planning entity has one or more planning variables. In case of N-Queen problem, Queen class is planning entity because it changes its row position.

\paragraph{Planning variable}
Planning variable is property of planning entity class with necessery getter and setter. In case of N-queen problem, row property is planning variable. It must be marked with @PlanningVariable annotation, which contains valueRangeProviderRefs property. This property defines which possible values of planning variable can be.


\paragraph{Planning value and planning value ranges}
Planning value is a possible value of a planning variable. Usually, a planning value is a problem fact but it can be any object for example a double. Planning value range is set of a possible planning values of planning variable. This set can be a countable (for example row 1, 2, 3 or 4) or uncountable (for example any double between 0.0 and 1.0). Value Range is markerd with @ValueRangeProvider annotation, which has property id pointing to valueRangeProviderRefs @PlanningVariable's property. This annotation can be located on 2 types of methods -- on the Solution and on the planning entity. Usually, first type is used. Also return type of the method can be 2 types -- collection or value range.

\paragraph{Planning problem and planning solution}
Each planning problem has to be packaged as a class which a solver uses to solve the problem. In the case of n-Queens problem, class must contains column, row, and queen lists. The planning problem corresponds with unresolved planning solution. The solution must be described by a class that implements the Solution interface. This interface requires to implement setScore, getScore and getProblemFacts methods.

\subsection{Solver configuration}
The second part of OptaPlanner configuration is configuration of a solver. It is described by XML file and it also can be changed dynamically at runtime using SolverConfig API. Basically, it can be divided into three parts:

\begin{enumerate}
\item \textbf{Model definion} -- consists of the name of a class that implements the solution interface and a name of class that represents the planning entity
\item \textbf{Score function definition} -- consists of settings such as a type of score and a class which calculate the score (or DRL file with a rules that are used for calculating)
\item \textbf{Optimization algorithms definition} -- contains settings of algorithms that are used to optimize the calculation for obtaining a best score of the problem
\end{enumerate}

Thanks to the score calculation, all of the initialized solution can be evaluated by a score and these are three following ways how it can be implemented:

\begin{itemize}
\item \textbf{Easy Java score calculation} -- a class that must implements the EasyScoreCalculator interface. The calculation is performed using single method, which should returns score of solution. This simple way how to calculate score is slower and less scalable than other methods.
\item \textbf{Incremental Java score calculation} -- a class that must implements the IncrementalScoreCalculator interface. The calculation is performed using several specific computational methods. This is a quicker approach, but more difficult for implementation.
\item \textbf{Drools score calculation} -- The calculation is performed using DRL rules. These are rules are stored in .drl file. More information about rules can be found on \cite{DroolsBook}. This approach is well optimizable but DRL language must be used.
\end{itemize}

\subsection{Loading of problem data set}
Last step before the start is the loading of problem data set into a solver. This is done by uploading planning entities and planning values to the appropriate collection in a class that implements an interface Solution. From this step, everything is ready for Solver start.

In N-Queen example, all the queens and all the rows have to be initialized and uploaded to the appropriate collections of NQueens solution class.

\subsection{Starting of Solver and acquiring the best solution}
Starting solver takes place simply by calling the solve() method of instance of the Solver class with the parameter containing reference to an instance of a class that implements the interface solution.

After the calculation by calling getBestSolution() method of a solver instace, the best solution is returned. In the case of N-Queen problem we should get a solution where each queen has assigned one row and if optimal solution is found no two queens are threaten to each other.

\paragraph{Termination}
Not all calculations terminate automatically and therefore it is sometimes necessary to add conditions which causes the end. Options that can stop a calculation are the following:

\begin{itemize}
\item \textbf{Time limit termination} -- occurs after exceeding the time limit. The range is from milliseconds to hours
\item \textbf{Best score termination} -- terminates when a specified best score is achieved
\item \textbf{Step count termination} -- occurs after exceeding the limit of step count of calculation
\item \textbf{Combining of multiple terminations} -- previous termination methods can be also combined
\item \textbf{Asynchronous termination from another thread} -- can be used if it is necessary to terminate the calculation differently than automated methods
\end{itemize}



\chapter{Porting of OptaPlanner to Android}\label{PortingChapter}
This chapter deals with differences between Java SE API and Android API. Although Java libraries of Andoid API are based on Java SE, a lot of libraries are missing or they are incomplete. It causes problems that are dealt with in this chapter. These libraries must be added to Android project or dependencies to libraries have to be removed from the OptaPlanner core.

Portation is modification of software for the purpose of usage on different platforms. Optaplanner is designed for Java SE platform and for integration on Android platform it is necessary to check the API to find what difference are and solve the problems that occurs.

In the first Section \ref{comparsion}, Java SE API and Android API are compared. There are also shown packages which are used by OptaPlanner and JavaBeans problem is introduces. The proposals for solution of JavaBean problem are described in Section \ref{JavaBeans} and its summary is presented in the last Section~\ref{summary}.

\section{Java core packages comparsion}\label{comparsion}
In this section, Java core packages of Java Standart Edition API are compared with packages of Android API. In second part of this section, it is pointed on packages which OptaPlanner tool uses.

\subsubsection{Java SE API and Android API}\label{apis}
Android API is based on Apache Harmony Java SE \cite{Apache} (the open source version of Java SE) but as can be seen in Table~\ref{javaDiff} it is not complete Java SE API. Out of 38 Java SE API packages, 20 packages are completely missing and 9 packages are incomplete in Android API

Packages for the graphical user interface such as \texttt{java.swing} and \texttt{java.awt} were replaced with the Android graphical elements. Graphical user interface is often separated from the computational part of the application therefore this is not a major problem for porting libraries and tools.

The real problem occurs when applications requires core classes that are not in the Android API.

\begin {table}[h!]
\begin{tabular}{|l|c|c|}
\hline
{\bf Java 6 SE Package} & {\bf Included in Android API} & {\bf Needed by OptaPlanner} \\
\hline \hline
java.applet           & No -- missing completely  & No\\
java.awt              & Yes -- incomplete         & No\\
java.beans            & Yes -- incomplete         & Yes\\
java.io               & Yes -- complete           & Yes\\
java.lang             & Yes -- incomplete         & Yes\\
java.math             & Yes -- complete           & Yes\\
java.net              & Yes -- complete           & Yes\\
java.nio              & Yes -- complete           & No\\
java.rmi              & No -- missing completely  & No\\
java.security         & Yes -- incomplete         & No\\
java.sql              & Yes -- complete           & No\\
java.text             & Yes -- complete           & No\\
java.util             & Yes -- incomplete         & Yes\\
javax.accessibility   & No -- missing completely  & No\\
javax.activation      & No -- missing completely  & No\\
javax.activity        & No -- missing completely  & No\\
javax.annotation      & No -- missing completely  & No\\
javax.crypto          & Yes -- complete           & No\\
javax.imageio         & No -- missing completely  & No\\
javax.jws             & No -- missing completely  & No\\
javax.lang            & No -- missing completely  & No\\
javax.management      & No -- missing completely  & No\\
javax.naming          & No -- missing completely  & No\\
javax.net             & Yes -- complete           & No\\
javax.print           & No -- missing completely  & No\\
javax.rmi             & No -- missing completely  & No\\
javax.script          & No -- missing completely  & No\\
javax.security        & Yes -- incomplete         & No\\
javax.sound           & No -- missing completely  & No\\
javax.sql             & Yes -- incomplete         & No\\
javax.swing           & No -- missing completely  & No\\
javax.tools           & No -- missing completely  & No\\
javax.transaction     & No -- missing completely  & No\\
javax.xml             & Yes -- incomplete         & No\\
org.ietf.jgss         & No -- missing completely  & No\\
org.omg               & No -- missing completely  & No\\
org.w3c.dom           & Yes -- incomplete         & No\\
org.xml.sax           & Yes -- complete           & No\\
\hline
\end{tabular}
\centering
\caption{Java 6 SE API packages in Android API and Optaplanner core}
\label{javaDiff}
\end{table}

\subsubsection{Android API and Optaplanner}
Table~\ref{javaDiff} shows that OptaPlanner tool directly uses only 6 of the total number of 38 packages of Java SE API, namely: \texttt{java.beans}, \texttt{java.io}, \texttt{java.lang}, \texttt{java.math}, \texttt{java.net} and \texttt{java.util}. Three of the used packages are incomplete but only one of them affects OptaPlanner and that is \texttt{java.beans} package. Possible solutions of this problem are described in the Section~\ref{JavaBeans}.
\paragraph{Java Beans}
Beans are reusable software technology that can be assembled to create other application. Mostly, its classes are used for creation of graphical user interface but they also can be used for event handling. More information about JavaBeans can be found in~\cite{Beans}.

\section{JavaBeans problem}\label{JavaBeans}
OptaPlanner is completely written in the Java language and one of its dependencies is package. As can be seen in the Table~\ref{javaDiff}, this package is incomplete and when a simple OptaPlanner project on Android platform is started, \texttt{ClassNotFoundException} is thrown because the necessary classes are not found in this package. In this section, the possible solutions of this problem are presented.

\subsection{Repacking of JavaBeans redistribution to Java core namespace}
The first of the ways how to replace the missing \texttt{java.beans} package is use of JavaBeans redistribution. These libraries are specially designed for the Android platform to support JavaBeans or other missing packages. If the OptaPlanner code should stay the same, it is necessary to repackage these libraries to Java core namespace. In the following paragraphs, two redistribution and the Jar Jar Links tool for repacking jar files are introduced. Last paragraph shows how solve problem with addition of Java core libraries to an Android project. 

\paragraph{OpenBeans}
OpenBeans project \cite{OpenBeans} is a redistribution of \texttt{java.beans} package based on the Apache Harmony project. It was created because of missing JavaBeans on the Android platform. Used namespace of this redistribution is \texttt{com.googlecode.openbeans}. OpenBeans is an open-source project and it is distributed as jar package that can be included into an Android project. 

\paragraph{Mad Robot}
A similar project to OpenBeans is called Mad Robot \cite{MadRobot}. As well as OpenBeans, it contains redistribution of \texttt{java.beans} package in \texttt{com.madrobot.beans} namespace but it also includes some additional packeges for database, graphic or geometry manipulation. This project is distributed as Maven dependency.
\\
\begin{lstlisting}[captionpos={b},caption={Command for repacking openbeans.jar file},frame={lines},label={command},basicstyle=\footnotesize]
java -jar jarjar.jar process rule.txt openbeans.jar javabeans.jar
\end{lstlisting}

\paragraph{Jar Jar Links}
Jar Jar Links \cite{JarJar} is a utility for repackaging Java libraries. It enables to repack Java classes from one namespace to another. It is necessary to include rules file which describe way how jar file should be repacked. Example of an one rule can be seen on Listing \ref{rule}. Finally, the command with three parameters: rule file, input jar and output jar has to be run to repack (Listing \ref{command}). 
\\
\begin{lstlisting}[captionpos={b},caption={Jar Jar Links rule for repacking OpenBeans to Java core namespace},frame={lines},label={rule},basicstyle=\footnotesize]
rule com.googlecode.openbeans.** java.beans.@1
\end{lstlisting}

\paragraph{Core library flag}
Translation of Android project that contains a class from namespace \texttt{java.*} or \texttt{javax.*} crashes during the translation which is highlighted by message about using of a classes from Java core namespace. This can be avoided by using the \texttt{--core-library} flag in the tool \texttt{dx} program which is located in android-sdk tools build folder. Adding a flag on the last line allows translation of application. Listing \ref{lastline} shows how this line should looks like.
\\
\begin{lstlisting}[captionpos={b},caption={Spanning tree broadcast algorithm.},frame={lines},label={lastline},basicstyle=\footnotesize]
exec java $javaOpts -jar "$jarpath" --core-library "$@"
\end{lstlisting}

\subsection{Use of the OpenJDK distribution source code}
This solution is based on the use of available source code of OpenJDK Java SE \cite{OpenJDK} which is an open-source distribution of Java SE. By adding sources to an Android project, it is possible to get the necessary libraries. The advantage of this solution is that the dependency failure are seen in translation and not when the application runs. This makes possible to choose only the required classes. However, this adjustment is not trivial. It needs to be done by a special tool that removes unused dependencies or it must be done manually.

\subsection{Use of pruned rt.jar from OpenJDK distribution}
The last option without interference to the source code is use of the rt.jar package which is part of the Java SE libraries. This package contains JavaBeans compiled classes and other parts of Java SE. Due to its size, it is not well suited for an Android applications and it also includes libraries that are already contained in Android API and it can causes collisions. Therefore, it has to be pruned. The advantage of pruning is that there is no need to worry about dependencies that are not needed for Optaplanner tool because these files are not again compiled. On the other hand, it may happen that an application hits some missing required dependencies during runtime and the application crashes.

\subsection{Use of OpenBeans in OptaPlanner project}
This is the first option which intervenes to the Optaplanner source code and it consists of replacing all \texttt{java.beans} dependencies for the \texttt{com.googlecode.openbeans} by rewriting all imports and by addition of OpenBeans.jar archive to the Optaplanner core project. This causes that all dependencies are redirected to OpenBeans. The disadvantage of this solution is the intervention to OptaPlanner source code. In terms of Android application developers, it is needed to create a new fork of OptaPlanner and modify it and this causes that the maintenance is then considerably complicated.

\subsection{Removing and replacing JavaBeans from OptaPlanner}
Last option to solve the JavaBeans problem is its elimination from the source code and its replacing by another technology. This is the biggest intervention to Optaplanner code of the offered solutions and it is also the major disadvantage.

\section{Summary of approaches}\label{summary}
Table \ref{advDis} shows the advantages and disadvantages of each proposals. Furthermore, there are mentioned licenses which should be respected when specific solution is chosen and there is also mentioned the each approach level of difficulty.

The best solution of JavaBeans problem seems to be repacking of the OpenBeans redistribution of JavaBean to Java core namespace. In this approach, Apache licence must be respected. This licence is free software license and it allows easily use the code. It is not necessary to modify the OptaPlaner code and generally, this approach requires less effort from the programmer.

\begin {table}[h!]
\begin{tabular}{|p{2.5cm}|p{2cm}|p{2.4cm}|p{2.1cm}|p{5cm}|}
\hline
{\bf Approach} & {\bf Licence} & {\bf Optaplanner modification} & {\bf Level of difficulty} & {\bf Advantages and disadvantages} \\
\hline \hline
    Repacking OpenBeans redistribution to Java core namespace & Apache License 2.0 & No & Easy & 
    {\bf \texttt{+}} standalone jar file, no problems with dependencies \\
    \hline
    Repacking of Mad Robot redistribution to Java core namespace & LGPL 2.1 & No & Easy &
    {\bf \texttt{+}} same as in previous case \\
\hline
    Use the OpenJDK distribution source code &  GPL 2.0 & No & Medium &
    {\bf \texttt{+}} dependency failure occurs in translation, source code control

    {\bf \texttt{-}} difficult adjustment which can cause problems with dependencies \\
\hline
    Use of pruned rt.jar from OpenJDK distribution & GPL 2.0  & No & Hard &
    {\bf \texttt{+}} standalone jar file 

    {\bf \texttt{-}}
    difficult adjustment which can cause problems with dependencies, incosistent jar, prunning \\
\hline
    Use of OpenBeans in OptaPlanner project & Apache License 2.0 & Yes & Easy &
    {\bf \texttt{+}} easy adjustment

    {\bf \texttt{-}} need of modification of Optaplanner source code and the subsequent maintenance of OptaPlanner fork \\
\hline
    Removing and replacing JavaBeans from OptaPlanner & -- & Yes & Medium &
    {\bf \texttt{-}} same as in previous case\\
\hline
\end{tabular}
\centering
\caption{Advantages and disadvantages of solutions of JavaBeans problem}
\label{advDis}
\end{table}




\chapter{Vehicle Routing Problem Application}\label{ApplicationChapter}
One of the goals of this thesis is creating of a~simple application which demonstrates functionality of OptaPlanner tool
on Android system. Previous Chapter~\ref{PortingChapter} shows how to port the tool to the mobile platform. In this
chapter, OptaPlanner tool is used to creating the Vehicle Routing Problem application.

First Section~\ref{RequirementsApplicationSection} introduces application requirements. Application design is described
in the second Section~\ref{ApplicationDesignSection}. Implementation itself is divided into two parts. The first part
which is described in Section~\ref{ApplicationImplementationSection} present inner structure of the application and the
second part deals with graphical user interface and its layout.

\section{Requirements for Android application}\label{RequirementsApplicationSection}
This section deals with requirements for Android application. First part introduces features of the application and the
second part discusses support for different versions of the Android operating system.

\subsection{Application features}\label{FeaturesSection}
The following paragraphs describe essential requirements and features of created application. They focus on inner
structure, graphical user interface and the licence under which the application is written.

\paragraph{Vehicle Routing application}
Standard OptaPlanner distribution~\cite{OptaPlannerDistribution} contains demonstration examples. One of the examples is
Vehicle Routing application. It is often used for presentation of OptaPlanner and it is one of the real world examples
and thus it is also a~good choice for demonstration on Android. Created aplication should present the Vehicle Routing
Problem in a~similar way as the original OptaPlanner application.

\paragraph{Vehicle Routing model}
The source code of the original application already contains OptaPlanner Vehicle Routing Problem model which should be
included in this application. Furthermore, it contains some tools for importing specific \texttt{.vrp} files. The model
specifies classes which are required for OptaPlanner tool.

\paragraph{Graphical user interface}
Graphical user interface cannot be ported because application is written by Awt and Swing libraries which are not
included in Android API as described in Section~\ref{ComparsionSection}. Therefore, new application graphical user
interface should be created and adjusted to fulfill aspects of Android application development.

\paragraph{Application settings}
Application without any settings is too static and uninteresting for the users. Therefore, they should at least be able
to choose problem solving algorithm. Another option could be setting of calculation time limit. Thanks to that, process
can be terminated earlier.

\paragraph{File opening}
The original Vehicle Routing application contains \texttt{.vrp} example files which contains tasks of problems. These
files should be compatible with created application and some of them should be included. It should be possible to open
these files and display them in a~similar way as in the original application.

\paragraph{Solution displaying}
The application must be able to display unsolved solution on the application screen. Furthermore, it should be possible
to display new best solution everytime when it is found and at the end of the process, last best solution should remain
on the screen. The application should also be able to display actual statistic information about solved problem.

\paragraph{License}
The entire application should be distributed as an~open-source software and it should be written under Apache License
2.0. Source code must be publicly available on the Internet for guidance of other people in their own OptaPlanner
projects.

\subsection{Android devices support}
Before the development starts, it is good to clarify which version of Android will be supported by application. Every
version of Android comes with new API and new functionality. Biggest changes comes when the first number of version is
changed. Actual distributions of Android versions on devices can be seen in the last column of
Table~\ref{AndroidHistoryTable}.

Versions 2.x.x are on decline. Currently, the most used versions are with 4.x designation and distribution of the newest
5.x versions grows. Therefore, it is decided that application should support Android from version 4.0 (API 14). The
version decides which resource can be used for application design and development and how application should be tested.

\section{Application design}\label{ApplicationDesignSection}
One of the critical points of creating an~application is design. It can be divided into two parts. The first one is
design of an~inner structure and it is presented in Section~\ref{ApplicationFeatureSection}. The second one is design
of graphical user interface and application component layout which is described in Section~\ref{ScreenDesignSection}.

\subsection{Application features}\label{ApplicationFeatureSection}
In Section~\ref{FeaturesSection}, inner structure requirements of the application are described. According to them, the
application features design is written and introduced in following paragraphs.

\paragraph{Application settings}
Designed application supports several setting options. It is possible to select one of three algorithms: First fit
decreasing with Late acceptance, Branch and bound and Brute force. Simultaneously, it is required to select time limit
of calculation in seconds. Calculation stops after time limit is reached or it is possible to stop it earlier by stop
button. Furthermore, Vehicle Routing example can be chosen. Application contains list of these files and after user
selects one of them everything should be prepared to calculation start. The setting should be placed on the first screen
of the application.

\paragraph{VRP files}
Mentioned list of example files consists from files used in original OptaPlanner application. These files are named the
same way and user can compare results between both applications. Because default Android system does not contain file
browser and user probably does not have his own \texttt{.vrp} files, application does not support opening files from
device storage. List of files should be placed on the second screen of the application.

\paragraph{Porting of Vehicle Routing Problem model}
In Section~\ref{FeaturesSection}, it is described that original application already contains Vehicle Routing model for
OptaPlanner tool. These classes have to be embedded into application and they have to be used for calculation of the
problem. Furthermore, they should be used for displaying of current solution.

\paragraph{Problem solving}
Graphical user interface cannot wait until calculation is finished and therefore these two parts must be separated from
each other. While solving is in progress, user controls the application and also can use some of its parts. When screen
is turned or application is hidden in background, solving process has to be still active and not terminated.

\paragraph{Solution displaying}
New best solution should be displayed every time when it is found. For that purpose, application contains third screen
especially for displaying founded solution which should be similar to original application. Lines represents roads,
vehicles and depots have their own icons. Customers differs from the original for saving screen space. Instead of points
with numbers, customers should be represented as circles with inner number describing its demand. Time window circles
should be displayed in a~similar way as the original but not separated from customer.

\paragraph{Solution data}
Every solution has data which cannot be displayed graphically or they are too important and have to be be displayed
separately. Hence, it is designed that score of solution and actual load of vehicles should be drawn on side menu which
is described in Section~\ref{ScreenDesignSection}.

\begin{figure}[h!]
    \centering
    \includegraphics[scale=0.7]{fig/sceen_design.pdf}
    \caption{Design of screens and links among them.}
    \label{ScreenDesignFigure}
\end{figure}

\subsection{Design of screens}\label{ScreenDesignSection}
This section introduces design of application screens and describes component layouts. The overall concept and links
between screens are shown in Figure~\ref{ScreenDesignFigure}.

\paragraph{Application top bar}
Top bar is placed on the top of the each screen as shown in Figure~\ref{ScreenDesignFigure}. It contains name of the
application and quick function buttons. These buttons can start the calculation or call the informational dialogs.
Buttons are not visible all the time but only on the screens when it is necessary.

\paragraph{Settings screen}
The first screen of Figure~\ref{ScreenDesignFigure} is main screen of the application. It consists of top bar, welcome
text and part with setting elements where calculation options can be set. Last element on the screen is button for
switching to the next screen where one of the VRP files can be selected.

\paragraph{Screen with VRP files list}
The second screen in Figure~\ref{ScreenDesignFigure} contains only list of VRP files and top bar. Top bar does not
contains any buttons on this screen because there is no need for them. The list contains all of the included VRP files
and after click on one of them, it is switched to the last screen -- Solution screen.

\paragraph{Solution screen}
The most important screen of the application is the last screen. Solution and its gradual progress is displayed on this
screen. On the top bar, button for start of calculation is included and under the actual solution representation,
progress bar for displaying actual time is placed. When user swipe with finger from left to right on the screen, side
menu with actual solution statistic should be displayed.

\paragraph{Side menu with statistics}
The last section of Figure~\ref{ScreenDesignFigure} is side menu which is displayed on right side on the solution
screen. It contains statistic data of actual solution. First item on the menu is solution score and next items
represents every vehicle of the problem and its current load and capacity. Menu can be closed when user clicks somewhere
outside of the menu.

\paragraph{Informational dialogs}
Application design contains two informational dialogs for better understanding of the application content. First one
contains information about application itself. Second one consists of legend which describes all displayed components of
Vehicle Routing Problem on the screen.

\paragraph{Material design}
One of the new features which Android version 5 brings is material design. It is very sophisticated study that shows how
to handle elements, layouts, colors and others. Although this feature is not fully backward compatible, it is partially
possible to bring this design to earlier devices with older versions of Android. It is designed that application should
use material design as much as possible.

\section{Application implementation}\label{ApplicationImplementationSection}
In this section, implementation of Vehicle Routing Problem application is described. The first part introduces
application structure and its important components. Second part focuses on porting of Vehicle Routing Problem and its
model from the original OptaPlanner application. Graphical user interface is presented in the next
Section~\ref{GuiSection}.

\subsection{Application structure}
Application consists of one activity and three fragments as shown in Figure~\ref{ActivityFragmentsFigure}. Activity is
represented by \texttt{MainActivity} class in the application code and it contains Action bar and space where fragments
are placed. Every time when action of fragment change is invoked, the space is replaced by new fragment.

Settings elements are placed on the first fragment which is defined by \texttt{MainFragment} class.
\texttt{VrpFileListFragment} class represent the second fragment containing list of VRP files. The last fragment is
defined by \texttt{VrpFragment} class and it is used for displaying current solution.

\begin{figure}[h!]
    \centering
    \includegraphics[scale=0.7]{fig/act_frag.pdf}
    \caption{Activity and fragments in application.}
    \label{ActivityFragmentsFigure}
\end{figure}

In the following paragraphs, important components of application are described. Especially, the components which are
related to the background processes. Graphical components and their layout are described in Section~\ref{GuiSection}.

\paragraph{List of files}
List of VRP example files is placed on the second fragment. This list is implemented by \texttt{RecyclerView} component
which simplifies displaying of data to the list and provides basic patterns of behavior.

\paragraph{Solver asynchronous process}
After the button for calculation start is pressed, asynchronous process is created. This process sets, builds and
activates Solver with required parameters. Listener is added to Solver to publish process every time when new best
solution is found. The process is represented by \texttt{VrpSolverTask} class which extends \texttt{AsyncTask}.
\texttt{AsyncTask} class enables changes of graphical user interface, perform the background operations and publishes
results.

\paragraph{Solution painter}
Solution painter is a~component which draws a~solutions on the screen. It is represented by \texttt{VrpPainter} class
which is modified \texttt{VehicleRoutingSolutionPainter} class from the original OptaPlanner project. Because Android
does not support Awt and Swing Java graphic libraries, \texttt{VehicleRoutingSolutionPainter} was rewritten to use
Android methods for drawing on the screen. Solution painter draws two types of solution:

\begin{enumerate}
    \item \textbf{Unsolved solution} --  painted when a~file is selected from the list in the second fragment.
    \item \textbf{New best solution} -- painted every time after Solver is activated and new best solution is found.
\end{enumerate}

\subsection{Porting of Vehicle Routing Problem}
Original Vehicle Routing application contains Vehicle Routing Problem model for OptaPlanner tool. These files are taken
and modified to fit in the created Android application. Following paragraphs present these files and show the way how
they are used.

\paragraph{Vehicle Routing Problem model}
Without model of the problem, application cannot work. Vehicle Routing Problem is defined by
\texttt{VehicleRoutingSolution} class which implements \texttt{Solution} interface. This class contains all information
about solved problem (list of all customers, depots and vehicles) and it is used together with solver configuration
by Solver to calculation of the problem. \texttt{Customer} class is marked as planned entity and contains
\texttt{Standstill} planning variable. More detailed information about Vehicle Routing Problem definition can be found
in OptaPlanner documentation~\cite{OptaPlannerDoc}.

\paragraph{Solver configurations}
In this application, it is possible to use three algorithms and set time limit of calculation. These configurations
include link to problem definition and link to score calculator and they are stored in \texttt{.xml} file which are used
for building the Solver. For each algorithm, there is one XML file and it is applied according to the choice in the
application. Time limit is additionally set after Solver is created.

\paragraph{Score calculator}
Every solution has its own score and this score must be calculated by one of the three methods described in
Section~\ref{ScoreConfigSection}. For score calculation in this application, \texttt{VehicleRoutingEasyScoreCalculator}
class which implements \texttt{EasyScoreCalculator} is used. This class calculates hard and soft score of solution. Hard
score is computed as a~load of vehicles above their capacity and soft score is calculated as negative total vehicle
distance. In case of time window variant, delay against due time of arrival is added to the hard score.

\paragraph{VRP example files}
Example \texttt{.vrp} files are used as problem datasets. These files are taken from the original OptaPlanner Vehicle
Routing Problem application. They contain information about number and capacity of vehicles, position and demands of
customers and position of depot. This application includes 36 example files in total.

\paragraph{Vehicle Routing importer}
Example files are stored in specific \texttt{.vrp} text format and have to be loaded into classes that describe Vehicle
Routing problem. For this purpose, \texttt{VehicleRoutingImporter} class is imported and used from the original
OptaPlanner application.

\section{Graphical user interface}\label{GuiSection}
This section presents graphical user interface implementation of the application as designed in
Section~\ref{ScreenDesignSection}. The first part of this section describes application screens and the second part
introduces three important components.

\subsection{Application screens}
Every application consists of Fragments or Activities which are collectively called screens. Using controls, it is
possible to move from one screen to another or to change its appearance or behavior. This application is composed from
three screens. These screens can be seen in Figure~\ref{ApplicationScreensFigure}. The third screen is displayed with
unsolved problem and with ongoing solution process.

\paragraph{Main screen}
Main screen is displayed after the application starts. It consists from Action bar, welcome text, setting elements
and button to continue to another screen. Action bar contains application name and buttons for displaying legend dialog
and application information dialog. Welcome text provides some basic instruction for the users. Two controls are present
for calculation options of the problem. It is possible to set time limit in seconds using Number picker and Spinner
allows to select one of the three supported algorithms. Last element on this screen is Open file button which opens
screen with list of VRP files.

\paragraph{Screen with VRP files list} After Open file button on the main screen is pressed, screen with VRP files is
displayed. It also contains action bar but it is very limited because no controls are required on this screen. Rest of
area is filled with list of VRP files from original OptaPlanner application~\cite{OptaPlannerDistribution}. After the
selected file is pressed, it is switched to last screen and the problem with its solutions is displayed.

\paragraph{Solution screen}
This screen is used to display unsolved, ongoing and solved solutions. It contains Action bar with all of the control
items as shown in Figure~\ref{ApplicationScreensFigure}. Compared to the main screen, Action bar has an~additionally
button for displaying Navigation drawer and button for start and end of the solution process. On the bottom of the
screen, Progress bar is placed. This component is used for displaying time which approximately remains. Rest of the
screen is filled with component which draws current solution. At the beginning, the component draws unsolved solution
and after the start button is pressed, it always draws the best solution after it is found. Individual elements of this
component are:

\begin{itemize}
  \item \textbf{Circle with a~number} -- customer with his demand.
  \item \textbf{Building image} -- depot from where vehicles depart.
  \item \textbf{Car image} -- vehicle with its color.
  \item \textbf{Solid line} -- vehicle road to a~customer.
  \item \textbf{Dashed line} -- vehicle road to a~depot.
  \item \textbf{Sector on a~circle} -- time windows for vehicle arrival.
  \item \textbf{Line on a~circle} -- vehicle arrival time.
\end{itemize}

\begin{figure}[h!]
    \centering
    \includegraphics[scale=0.15]{fig/screens.png}
    \caption{Application screens -- main screen, list of VRP files, screen with unsolved problem and ongoing solution
    process.}
    \label{ApplicationScreensFigure}
\end{figure}

\subsection{Application components}
This section introduces and describes three important components of the application: Action bar, Navigation drawer and
Dialogs.

\subsubsection{Action bar}
Action bar is a~panel on top of the screen that provides basic user action and information about user navigation. It
always contains application name, optionally action buttons for quick invocation of application functionality and
overflow button on the right side for displaying the other applications options.

Action bar is displayed on every screen of this application but it changes depending on required functions on actual
screen. Figure~\ref{ApplicationScreensFigure} shows action bars of each screen.

\subsubsection{Navigation drawer}
Navigation drawer is a~panel that displays application navigation on the left edge of the screen. By default, it is
hidden and it could be displayed by touching the left icon on the action bar. Also, it could be displayed when a~user
swipes with a~finger from the left edge of the screen to the right. Opposite procedure makes navigation drawer
invisible.

This application uses navigation drawer for displaying important statistic data. Figure~\ref{NavigationDrawerFigure}
displays visible panel on the left side of the application. The first item on the panel shows hard and soft score of
displayed solution. Second item holds total distance traveled. Other items are linked to vehicles of the problem. Each
of them has its own parameters -- color, name and capacity. These three items are static and do not change during the
calculation. Last parameter is actual load of the vehicle.
\\
\begin{figure}[h!]
    \centering
    \includegraphics[scale=0.15]{fig/nav_drawer.png}
    \caption{Navigation drawer with actual data.}
    \label{NavigationDrawerFigure}
\end{figure}

\subsubsection{Dialogs}
Dialogs are small windows which display some significant information or they are used for user interaction with
decisions that define further actions. Dialogs are always located above all other parts of the application.

Figure~\ref{DialogsFigure} shows all three dialogs used in the application. The first one and the second one can be
retrieved directly from the action bar by clicking on the icons with question mark or informative icon. The first dialog
contains application legend for understanding what is displayed on the screen and the second dialog briefly describes
the application. The last dialog is displayed only when calculation runs and user clicks on the back button. Dialog then
asks the user if he wants to end the ongoing calculation.
\\
\begin{figure}[h!]
    \centering
    \includegraphics[scale=0.15]{fig/dialogs.png}
    \caption{Screenshots of dialogs used in the application.}
    \label{DialogsFigure}
\end{figure}


\chapter{Testing and future work}\label{TestingChapter}

\section{Testing}

\subsubsection{Devices}
The application was tested on several devices during and also after developement. Table \ref{TestingDevicesTable} shows
these testing devices and their important parameters. Some issues which are associated to them are described in
following paragraphs.

First one is Android version which specifies supported parts of Android features. Most of problems are related with
unsupported functions and they are ordinarily detected by compiler. However, from time to time some problems can appear
and therefore it is better to test application on various version of system.

Next parameters are display size and display resolution which together define density of screens points. Every device
has diferent display density and component layout must be well designed to fit on so many displays. Also after screen
rotation, components layout changes and it must be adapted.

Last parameters CPU and RAM defines device speed and computing capabilities. Because this application calculates with
many values, it was performed some tests and measurements which compare these devices.

\begin {table}[h!]
\begin{adjustwidth}{-1cm}{}
    \begin{tabular}{|l|c|c|c|c|}
        \hline
        \textbf{Device (Android version)} & \textbf{Display size} & \textbf{Display resolution} & \textbf{CPU} & \textbf{RAM} \\ \hline \hline
        LG Nexus 5 (5.1.0)            & 4.95 inches  & 1080 x 1920 pixels & 4 core 2.3 GHz & 2 GB   \\ \hline
        Asus Nexus 7 2013 (5.1.0)     & 7.0 inches   & 1200 x 1920 pixels & 4 core 1.5 GHz & 2 GB   \\ \hline
        Samsung Galaxy Xcover (4.1.2) & 4.0 inches   & 480 x 800 pixels   & 2 core 1 GHz   & 1 GB   \\ \hline
        Sony Xperia active (4.0.4)    & 3.0 inches   & 320 x 480 pixels   & 1 GHz          & 0.5 GB \\ \hline
    \end{tabular}
    \centering
    \caption{Testing devices}
    \label{TestingDevicesTable}
    \end{adjustwidth}
\end{table}

\subsubsection{Testing measurement}
After debugging the application, two measurements were performed on devices mentioned in Table \ref{TestingDevicesTable}
and one desktop computer. Parameters of the computer was:

\begin{itemize}
\item \textbf{Device} -- Notebook Lenovo ThinkPad T430s
\item \textbf{Processor} -- Intel Core i7-3520M 2.90GHz
\item \textbf{RAM} -- 16 GB
\item \textbf{OS} -- Fedora 21 64b
\end{itemize}

For both tests, three testing vrp file was used. They differ with number of customers and number and capacity of
vehicles. Parameters of these files are shown in table \ref{TestingFilesTable}. All the cases was measured five times
and it was created diameter of values in case of first measurement or best reached score was used in case of second
measurement.

\begin {table}[h!]
    \begin{tabular}{|l|c|c|c|}
        \hline
        \textbf{Vrp file} & \textbf{Number of costumers} & \textbf{Number of vehicles} & \textbf{Vehicle capacity} \\ \hline \hline
        A-n32-k5.vrp      & 32   & 5 & 100   \\ \hline
        A-n64-k9.vrp      & 64   & 9 & 100   \\ \hline
        F-n135-k7.vrp     & 135  & 7 & 2210  \\ \hline
    \end{tabular}
    \centering
    \caption{Testing files}
    \label{TestingFilesTable}
\end{table}

First test dealt with time of first found solution. Every test was measure five times and it was created diameter which
is written in Table \ref{FirstFoundTable}. As can be seen, difference between desktop computer and mobile devices is
quite substantial and with more complicated problem difference between these devices increasing. It can also be observed
that more powerful mobile devices surpass older device with worse CPU.

\begin {table}[h!]
    \begin{tabular}{|l|c|c|c|}
        \hline
        \textbf{Device}       & \textbf{A-n32-k5.vrp} & \textbf{A-n64-k9.vrp} & \textbf{F-n135-k7.vrp} \\ \hline \hline
        LG Nexus 5            & 0,162 s               & 0,561 s               & 2,760 s                \\ \hline
        Asus Nexus 7 2013     & 0,228 s               & 0,795 s               & 4,132 s                \\ \hline
        Samsung Galaxy Xcover & 0,411 s               & 1,434 s               & 8,219 s                \\ \hline
        Sony Xperia active    & 0,755 s               & 2,705 s               & 16,334 s               \\ \hline
        Desktop computer      & 0,075 s               & 0,130 s               & 0,284 s                \\ \hline
    \end{tabular}
    \centering
    \caption{Time of first found solution}
    \label{FirstFoundTable}
\end{table}

Second test is displayed in Table \ref{ScoreLimitTable} and it was based on 10s time limit. After the time limit the
best soft score was saved and from five measurements the lowest was selected. Soft score shows distace which should be
traveled with all of the vehicles. Quite substantial difference is appeared again between mobile devices and desktop
computer. Although nexus devices have different parameters they reached same results in two cases. The oldest device
Sony Xperia active did not reach any result in that time limit for the last case.

\begin {table}[h!]
    \begin{tabular}{|l|c|c|c|}
        \hline
        \textbf{Device}       & \textbf{A-n32-k5.vrp} & \textbf{A-n64-k9.vrp} & \textbf{F-n135-k7.vrp} \\ \hline \hline
        LG Nexus 5            & 857311                & 1597400               & 1411795                \\ \hline
        Asus Nexus 7 2013     & 857311                & 1624700               & 1411795                \\ \hline
        Samsung Galaxy Xcover & 879018                & 1631923               & 1420843                \\ \hline
        Sony Xperia active    & 894553                & 1633708               & --                     \\ \hline
        Desktop computer      & 787082                & 1424148               & 1292057                \\ \hline
    \end{tabular}
    \centering
    \caption{Soft score after 10s limit}
    \label{ScoreLimitTable}
\end{table}

Although, there is significant difference between desktop computer and Android phones or tablet, these mobile devices
still sufficient fast to use OptaPlanner tool and solve such kind of problem.




\section{Future work}

\paragraph{OptaPlanner game}
Because OptaPlanner works on Android, following work which has already started is creating of simple game where user
goal is defeat the OptaPlanner by finding shortest way for vehicles. Using clicking on the customers user create road
for vehicle to the depot. Meanwhile time is measured and when user finish its work OptaPlanner do the same and results
will be compared. This is the early proposal of the game which follows this work.

\paragraph{Drools}
As was written in Chapter \ref{PortingChapter} Drools package was not included in poration due to its size and
complexity. There are plans to port Drools tool to Android separately and if portation is successful, it will be
deployed and tested together with OptaPlanner. Although the Drools tool is consuming lot of computer resources, current
mobile devices already have high performance and it might be interesting to have such instrument on Android platform.


\chapter{Conclusion}\label{ConclusionChapter}
In this project, the Java programing language, its most common platform Java Standard Edition, the Android operating system and the OptaPlanner tool were presented. Java is cross-platform language which is used for application development for Android operating system. Android is largely used as primary system for mobile devices and OptaPlanner is tool for solving optimalization problems. OptaPlanner is completely written in Java and it was not ported on Android yet. Therefore, this work focus on the portation of this tool and deals with a problems associated with it.

Application Programming Interfaces libraries of Java SE and Android API were compared and it was found that they differ significantly. Out of 38 Java SE API packages, 20 packages are completely missing and 9 packages are incomplete in Android API. On closer inspection, it was revealed that one of the OptaPlanner tool dependecies is missing in Android API. This missing package is named \texttt{java.beans} and it covers JavaBeans technology.

For JavaBeans problem were suggested five solutions, namely: Repacking JavaBeans redistribution to Java core namespace, Use of the OpenJDK distribution source code, Use of pruned rt.jar from OpenJDK distribution, Use of OpenBeans in OptaPlanner project and Removing and replacing JavaBeans from OptaPlanner. Due to comparison of advantages and disadvantages of individual solutions, it was selected a solution which does not change source code of Optaplanner tool, has a license that allows easy application and generally is suitable for application on Android.

The proposed solution for further progress is repacking of JavaBeans redistribution to Java core namespace. Next procedure will be portation of OptaPlanner core to Android platform and creating of simple Android application which demonstrates vehicle routing problem on ported tool. 

 % viz. obsah.tex

  % Pouzita literatura
  % ----------------------------------------------
\ifczech
  \bibliographystyle{czechiso}
\else 
  \bibliographystyle{plain}
%  \bibliographystyle{alpha}
\fi
  \begin{flushleft}
  \bibliography{literatura} % viz. literatura.bib
  \end{flushleft}
  \appendix
  
  \chapter{Content of the DVD}

%\chapter{Manual}
%\chapter{Konfigrační soubor}
%\chapter{RelaxNG Schéma konfiguračního soboru}
%\chapter{Plakat}

 % viz. prilohy.tex
\end{document}
