% todo introduction to proposal

\section{Requirements for OptaPlanner portation}
In chapter \ref{PortingChapter}, there was describe difference between Java SE API and Android API. Also there was explainded problem
with JavaBeans package which is necessery for OptaPlanner library. This package needs to be included into project because it is not contained
in Android API. From described solution of JavaBeans package replacement was selected first option: Repacking OpenBeans package into Java core
name space.

One of the requirement of portation should be automatic preparation of enviroment. When application is built it is good to make it all automatic.
JavaBeans library should be included or should be prepared by build script. Also application should be built without much efford from side of programmer.

As was described in chapter OptaPlanner, one of the way of calculation function in OptaPlanner is by the Drools rules. Drools is a big library and project
which allows to write prescriptions for calculation of best choises. Because this project is not good optimalized for mobile platforms it was not included
in the portation of OptaPlanner to Android. Instead of drools calculation standart Java calculation will be used as default tool.

Another problem which may occur is problem of user classes in core namespace. When application is built, compiler reports a problem with included classes.
It is excepted that java core name space is reserved for libraries from standart java api and not for user-created libraries. Therefor it is needed to add
--core-library flag to dx file in android-sdk.

In summary, it is necessery to repack OpenBeans to Java core namespace, include this package in application project, write automatic scripts which do all
the work and exclude drools package from project to prevent possible problems and generally decrease size of the application.

\subsubsection{Addition of JavaBeans package}
The OpenBeans package can be repackaged into Java core name space and can be ordinary included as standard jar file. Howewer, if new verision of OpenBeans
released it is good to make the process automatic. Basic proposal of repacking OpenBeans is following:
\begin{itemize}
\item Download JarJar tool for repacking libraries
\item Download OpenBeans package which is in com.googlecode namaspace
\item Run JarJar tools and repack OpenBeans package to java core namespace
\item Clean temporary files after process and move final jar into lib directory.
\item Add --core-library flag to dx file in android-sdk build tools
\end{itemize}

\section{Developer tools}
Google supports two developer tools for developing Android applications. From the very beginning, eclipse with Android plugin was primary environment built
on standarts of this system.

Android studio is new deleloper tool which bring many different features againts Eclipse. It is based on developer enviroment Idea J with plugins supporting
Android components. Also it brings new build system with base on gradle language.


\section{Requirements for Android application}



\subsubsection{Android devices support}
Before the development starts, it is good to clarify for which version of Android application will be supported. Every version of Android
comes with new API and new functionality. Biggest changes comes when first number of version is changed. Table \ref{distributon} shows
actual distributions of android verisons on devices.

Currently versions 2.x.x are on the decline, the most used version is 4.x and the newest version is still not so much used. Therefore it
was decided that application will supports android from version 4.0 (API 14).

\begin {table}[h!]
    \begin{tabular}{|c|c|r|}
        \hline
        Android version   & API   & Distribution  \\ \hline \hline
        2.2               & 8     & 0.4 \%        \\ \hline
        2.3.3 -- 2.3.7    & 10    & 6.4 \%        \\ \hline
        4.0.3 -- 4.0.4    & 15    & 5.7 \%        \\ \hline
        4.1               & 16    & 16.5 \%       \\ \hline
        4.2               & 17    & 18.6 \%       \\ \hline
        4.3               & 18    & 5.6 \%        \\ \hline
        4.4               & 19    & 41.4 \%       \\ \hline
        5.0               & 21    & 5.0 \%        \\ \hline
        5.1               & 22    & 0.4 \%        \\ \hline
    \end{tabular}
    \centering
    \caption{Android version distributon \cite{Dashboards}}
    \label{distributon}
\end{table}



\subsection{Design of screens}

\subsection{Functionality of application}

\section{Material design}
One of the new features which bring Android version 5 is material design. It is very sophisticated study that show how to handle with with elements,
layouts, colors and more. Although this property is not fully backward compatible, it is possible partially to bring this design to earlier devices
with older versions of Android.

This application uses material design as much as possible. Every screen contains material design elements which will be described bellow.

%diagram pruchodem programu nebo uml
