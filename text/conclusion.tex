Libraries of Java SE API and Android API were compared and it was found that they differ significantly. Out of 38 Java
SE API packages, 20 packages are completely missing and 9 packages are incomplete in Android API. On closer inspection,
it was revealed that one of the OptaPlanner tool dependencies is missing in Android API. This missing package is named
JavaBeans.

For JavaBeans problem were suggested five solutions, namely: Repacking JavaBeans redistribution to Java core namespace,
Use of the OpenJDK distribution source code, Use of pruned rt.jar from OpenJDK distribution, Use of OpenBeans in
OptaPlanner project and Removing and replacing JavaBeans from OptaPlanner. Due to comparison of advantages and
disadvantages of individual solutions, it was selected a~solution which does not change source code of OptaPlanner tool,
has a~license that allows easy application and generally is suitable for usage on Android.

Chosen solution for further progress consists of repacking of OpenBeans redistribution to Java core namespace. During
this process, new JAR file consisting of the missing JavaBeans libraries is created and could be subsequently inserted
into an~Android project. However, this faces the problem which does not allow to use classes of Java core namespace on
Android. The problem was resolved by using the Core library flag and the entire process of the portation was designed,
implemented and automated by Gradle language -- the default build tool for Android projects.

According to the written requirements, model Android Vehicle Routing Problem application which uses the ported
OptaPlanner was designed and implemented. The application can use one of the three algorithms, set calculation time
limit and solve attached problem examples. Actual problem is always displayed in text and graphic form on the screen of
the application together with the newest found solution. The application is publicly available on Google Play Store on
the Internet and the presentation video was created and was presented along with solution how to use OptaPlanner on
Android to developers community on OptaPlanner website.

On the Vehicle Routing Problem example, comparative measurements that focus on performance comparison between mobile
devices and desktop computer were performed. These measurements proved that OptaPlanner is working and can be used on
Android.

The challenge for the future work is portation of Drools tool. It is standalone project which can be used by OptaPlanner
as one of the option for calculation the score and in the future, it may be interesting to have such tool on Android.
