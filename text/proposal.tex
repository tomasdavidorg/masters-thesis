\section{Requirements for OptaPlanner portation}

\paragraph{Automatic preparation of enviroment}
First of the requirements of portation is automatic preparation of enviroment. When an application is built it is good
to make all automatic. OptaPlanner libaries should be correctly imported and all missing parts as JavaBeans library
should be included or should be prepared by build script. In summary, if the portation has special needs it must be done
by way which does not required much efford from side of programmer.

\paragraph{Drools library}
As was described in Chapter \ref{OptaPlannerChapter}, one of the ways of calculation function in OptaPlanner is by the
Drools rules. Drools is a big library and project which allows to write rule prescriptions for calculation of best
choices. Because this project is not optimalized for mobile platforms it should not be included in the portation of
OptaPlanner to Android. Instead of drools calculation standart Java calculation can be used as a default tool.

\paragraph{Adding of JavaBeans libraries}
The JavaBeans problem, which was mentioned in OptaPlaner portation analysis, should be also covered. Repacking of
JavaBeans redistribution to Java core namespace is the selected way how to solve that problem. It should be automatic
and all the necessary tools have to be downloaded from the internet (not included directly).

\paragraph{Core library flag}
This is the problem which is directly connected to the previous one. If libraries or classes from java core namespace
are added into Android project, compiler reports an problem and project is not correctly compiled until these files are
moved into another namespace. This can be avoided by using \texttt{--core-library} flag in dx file in android-sdk.
It is desirable that the flag is used and it also should be done by an automatic way.

\paragraph{Optaplanner usability}
In summary, it is necessery to prepare OptaPlanner for the Android platform to enable possibility of using tools for
solving planning problems. Also it should be aimed to OptaPlanner size to increasing applications speed.

\section{Requirements for Android application}

\subsection{Application features}

\paragraph{Vehicle Routing Problem}
Standard OptaPlanner distribution \cite{OptaPlannerDistribution} contains demonstration examples. One of the examples is
Vehicle Routing application. Its source code already contains Vehicle Routing Problem model which should be included in
this application. Also it contains some tools for importing specific \texttt{.vrp} files. Graphical user interface
cannot be ported because application is written by Awt and Swing libraries which are not included in Android API as
described in Section \ref{OptaPlannerChapter}. Thanks to that and because it is good practise to adapt the application
to Android mobile platform, graphical user interface and functions of application should be completely rewritten and
adjusted to fulfill aspects of Android application development.

\paragraph{Application settings}
Application without any settings is too static and user cannot do much with it. Therefore, it should be option to choose
problem solving algorithm  which can show that not all algorithms are suitable for such problems. Another option should
be setting of the time limit of caculation. Thanks to this, process will be terminated earlier and last solution remains
shown on the screen. Also it should be possible to terminate process before time limit.

\paragraph{File opening}
Original Vehicle Routing application contains \texttt{.vrp} example files. These files should be compatible with created
application and some of them should be included. Because Android system by default does not contain file browser and
user probably do not have own \texttt{.vrp} files application will not support opening files from device storage.

\paragraph{Solution diplaying}
When user selects file, unsolved solution should be displayable on the application screen. After start of solving
process, new best solution also should be displayable every time when new it is found. Application should further
contains resources how to display actual statistic information about solved problem.

\paragraph{License}
Whole application should be distributed as opensource software and it should be written under Apache License 2.0.
Source code must be publicly available on the internet to guide other persons how to use OptaPlaner in their projects.

\subsection{Android devices support}
Before the development starts, it is good to clarify for which version of Android will be application supported. Every
version of Android comes with new API and new functionality. Biggest changes comes when first number of version is
changed. Table \ref{distributon} shows actual distributions of Android verisons on devices.

Versions 2.x.x are on the decline, the most used versions are 4.x with higest percentage representation and the newest
version 5.x is not so much used yet but their participation grows. Therefore it was decided that application should
supports android from version 4.0 (API 14). It desides which resource can be used for application proposal and
development and how application should be tested.

\begin {table}[h!]
    \begin{tabular}{|c|c|r|}
        \hline
        Android version   & API   & Distribution  \\ \hline \hline
        2.2               & 8     & 0.4 \%        \\ \hline
        2.3.3 -- 2.3.7    & 10    & 6.4 \%        \\ \hline
        4.0.3 -- 4.0.4    & 15    & 5.7 \%        \\ \hline
        4.1               & 16    & 16.5 \%       \\ \hline
        4.2               & 17    & 18.6 \%       \\ \hline
        4.3               & 18    & 5.6 \%        \\ \hline
        4.4               & 19    & 41.4 \%       \\ \hline
        5.0               & 21    & 5.0 \%        \\ \hline
        5.1               & 22    & 0.4 \%        \\ \hline
    \end{tabular}
    \centering
    \caption{Android version distributon \cite{Dashboards}}
    \label{distributon}
\end{table}


\section{Portation proposal}

\paragraph{Developer tools}
Google supports two developer tools for developing Android applications. From the very beginning, Eclipse with Android
plugin was primary environment. Android studio is newer deleloper tool which bring many different features comparing to
Eclipse. It is based on enviroment IntelliJ Idea with plugins supporting Android features. Also it brings
new build system with base on Gradle language. This laguage will be also used for automatization of process desribed
bellow in this section.

\paragraph{OptaPlanner libraries import}
Due to Gradle scripts support Maven dependencies and it is just one of the distribution of Optaplanner, libraries can
be imported this way. Because OptaPlanner is daily developed, its last snapshot version will be used to secure that new
released version will be also compatible with Android.

\paragraph{Exclusion of unnecessary libraries}
Drools libraries will be excluded from portation. Because Gradle scripts are used, maven dependency to Optaplanner
library will contain this exclusion. This helps to lower size of created applications.

\paragraph{Completion of missing libraries}
JavaBeans library will be completed automaticly by Gradle scripts. Basic proposal of procedure of repacking OpenBeans is
following:
\begin{itemize}
\item Download JarJar tool for repacking libraries.
\item Download OpenBeans package which is in \texttt{com.googlecode} namespace.
\item Run JarJar tools and repack OpenBeans package to java core namespace (\texttt{java.beans}).
\item Clean temporary files and move final jar into lib directory.
\item Add \texttt{--core-library} flag to dx file in Android SDK build directory.
\end{itemize}

\paragraph{Adding of core library flag}
Before application is build, core library flag should be added into dx file Android SDK build tools directory. This
process have to be also automatic and performed by Gradle scripts. It should enter into proper directory find correct
file and adde flag to the right place.

% TODO ###################################################################
\section{Application proposal}

\subsection{Application features}

\paragraph{Application settings}

\paragraph{Vrp files}

\paragraph{Porting of Vehicle Routing Problem}

\paragraph{Solution solving}

\paragraph{Solution displaying}

\paragraph{Solution data}

\subsection{Design of screens}

\paragraph{Application top bar}

\paragraph{Settings screen}

\paragraph{Screen with vrp files list}

\paragraph{Screen of solution}

\paragraph{Side menu with statistics}

\paragraph{Informational dialogs}

\paragraph{Material design}
One of the new features which bring Android version 5 is material design. It is very sophisticated study that show how to handle with with elements,
layouts, colors and more. Although this property is not fully backward compatible, it is possible partially to bring this design to earlier devices
with older versions of Android.

This application uses material design as much as possible. Every screen contains material design elements which will be described bellow.
