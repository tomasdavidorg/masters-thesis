Testing of applications is very important part of development. During this phase, a large number of bugs is often
detected and critical problems can be also discovered. Measurement is as well counted to the testing because it can
reveal performance problems.

Future work shows direction of the continuation of the project and points to important parts which can be implemented or
improved.

Section \ref{TestingSection} focused on testing devices and describes performed comparative measurements. In the section
\ref{FutureWorkSection}, direction future work is discussed.

\section{Testing}\label{TestingSection}
This section introduces testing devices and describes performed comparative measurements. It focuses on comparison
between mobile devices and desktop computer which differ especially in their performance.

\subsubsection{Devices}
The application is tested on several devices during and also after development. Table \ref{TestingDevicesTable} shows
testing devices and their important parameters. Some issues which are associated to them are described in following
paragraphs.

First of the parameters is Android version which specifies supported parts of Android features. Most of problems are
related with unsupported functions and they are ordinarily detected by compiler. However, from time to time some
problems can appear and therefore it is better to test application on various versions of Android system.

Next parameters are display size and display resolution which together define density of the screens points. Every
device has diferent display density and component layout must be well designed to fit to many devices displays. Another
problem appears after screen rotation. Display sides are exchanded and components layout must be adapted.

Last parameters are number of CPU cores, CPU and RAM. They defines device speed and computing capabilities. Because this
application calculates with many values, it is performed some tests and measurements which compare these devices.

The application is tested on the five devices in total. Four of them are mobile phones, one is tablet and the last one
is Android emulator. Emulator is a virtual mobile device that runs on a desktop computer. Parameters of emulator can be
selected as necessary except of CPU which is inherited from host computer and adjusted according to operating system
capabilities.

\begin{table}[h!]
    \scalebox{0.9}{
        \begin{tabular}{|c|c|c|c|c|c|c|}
            \hline
            \textbf{Device} &
            \multicolumn{1}{p{1.5cm}|}{\centering \textbf{Android version}} &
            \multicolumn{1}{p{1.5cm}|}{\centering \textbf{Display size [inches]}} &
            \multicolumn{1}{p{2cm}|}{\centering \textbf{Display resolution [pixels]}} &
            \multicolumn{1}{p{1cm}|}{\centering \textbf{CPU cores}} &
            \multicolumn{1}{p{2cm}|}{\centering \textbf{\textbf{CPU [GHz]}}} &
            \multicolumn{1}{p{1.5cm}|}{\centering \textbf{\textbf{RAM [GB]}}} \\ \hline \hline
            LG Nexus 5             & 5.1.0 & 4.95 & 1080 x 1920 & 4   & 2.3 GHz & 2   \\ \hline
            Asus Nexus 7 2013      & 5.1.0 & 7.0  & 1200 x 1920 & 4   & 1.5 GHz & 2   \\ \hline
            Samsung Galaxy Xcover  & 4.1.2 & 4.0  & 480 x 800   & 2   & 1 GHz   & 1   \\ \hline
            Sony Xperia active     & 4.0.4 & 3.0  & 320 x 480   & 1   & 1 GHz   & 0.5 \\ \hline
            Emulator               & 4.4.2 & 4.5  & 720 x 1280  & --  & --      & 1.5 \\ \hline
        \end{tabular}
    }
    \centering
    \caption{Configurations of testing devices.}
    \label{TestingDevicesTable}
\end{table}

\subsubsection{Comparative measurements}
After testing and debugging the application, two comparative measurements are performed on the devices mentioned in
Table \ref{TestingDevicesTable} and one desktop computer with parameters described in Table \ref{TestingComputerTable}.
On the desktop computer, simple OptaPlanner project which calculates the same vehicle routing problems has been created.
On the mobile testing devices, graphical part is excluded and only calculation itself is measured. The goal of these
measurements is to compare capabilities of Android devices with a standard desktop computer and it should verify that
Android device are capable to use OptaPlanner.

\begin{table}[h!]
    \begin{tabular}{|c|c|}
        \hline
        \textbf{Parameter} &
        \textbf{Value} \\ \hline \hline
        Device    & Notebook Lenovo ThinkPad T430s \\ \hline
        CPU       & Intel Core i7-3520M 2.90GHz    \\ \hline
        RAM       & 16 GB                          \\ \hline
        OS        & Fedora 21 64b                  \\ \hline
    \end{tabular}
    \centering
    \caption{Configuration of testing computer.}
    \label{TestingComputerTable}
\end{table}

For both tests, three testing vrp files are used. They differ in the number of customers, the number of vehicles and
the capacity of one vehicle. Parameters of the three used files are shown in Table \ref{TestingFilesTable}. The same
files are also used in the application on the desktop computer.

\begin{table}[h!]
    \begin{tabular}{|l|c|c|c|}
        \hline
        \textbf{Vrp file} &
        \textbf{Number of costumers} &
        \textbf{Number of vehicles} &
        \textbf{Vehicle capacity} \\ \hline \hline
        A-n32-k5.vrp   & 31   & 5 & 100  \\ \hline
        A-n64-k9.vrp   & 63   & 9 & 100  \\ \hline
        F-n135-k7.vrp  & 134  & 7 & 2210 \\ \hline
    \end{tabular}
    \centering
    \caption{Testing files.}
    \label{TestingFilesTable}
\end{table}

First test deals with a time of the first found solution. Each test case is measured five times and it is calculated
arithmetic mean value ($\mu$) and standard deviation ($\sigma$). The values are written in Table \ref{FirstFoundTable}.
As can be seen, differences between the desktop computer and the mobile devices is quite substantial and with more
complicated problem difference between them rapidly grows. It can be also observed that more powerful mobile devices
(Nexus 5 and 7) surpass older devices (Samsung Xcover and Sony Xperia) with worse CPU and performance of emaulator is
approximately equal to the most powerful mobile device (Nexus 5).

\begin{table}[h!]
    \catcode`\-=12
    \begin{tabular}{|l|c|c|c|c|c|c|}
        \hline
        \multirow{2}{*}{\textbf{Device}} &
        \multicolumn{2}{c|}{\textbf{A-n32-k5.vrp}} &
        \multicolumn{2}{c|}{\textbf{A-n64-k9.vrp}} &
        \multicolumn{2}{c|}{\textbf{F-n135-k7.vrp}} \\ \cline{2-7}
        & $\mu$ & $\sigma$ & $\mu$ & $\sigma$ & $\mu$ & $\sigma$ \\ \hline \hline
        LG Nexus 5            & 0,162 & 0,034 & 0,561 & 0,031 & 2,760  & 0,063 \\ \hline
        Asus Nexus 7 2013     & 0,228 & 0,027 & 0,795 & 0,023 & 4,132  & 0,070 \\ \hline
        Samsung Galaxy Xcover & 0,411 & 0,058 & 1,434 & 0,012 & 8,219  & 0,217 \\ \hline
        Sony Xperia active    & 0,755 & 0,136 & 2,705 & 0,027 & 16,334 & 0,062 \\ \hline
        Emulator              & 0,119 & 0,021 & 0,386 & 0,056 & 1,647  & 0,015 \\ \hline
        Desktop computer      & 0,075 & 0,006 & 0,130 & 0,013 & 0,284  & 0,017 \\ \hline
    \end{tabular}
    \centering
    \caption{Time of first found solution in seconds (less is better).}
    \label{FirstFoundTable}
\end{table}

Second test is displayed in Table \ref{ScoreLimitTable} and it is based on 10 seconds time limit. After the time limit
the best soft score was saved and from five measurements the largest was selected. Soft score shows a distance of the
all vehicles as a negative value which is traveled betweeen customers and depot. A higher value means a smaller
distance. Quite substantial difference is appeared again between mobile devices and desktop computer. Although the Nexus
devices have different parameters they reached same results in two cases. The oldest device (Sony Xperia active) did not
reach any result in the appointed time limit for the last case.

\begin{table}[h!]
    \begin{tabular}{|l|c|c|c|}
        \hline
        \textbf{Device} &
        \textbf{A-n32-k5.vrp} &
        \textbf{A-n64-k9.vrp} &
        \textbf{F-n135-k7.vrp} \\ \hline \hline
        LG Nexus 5            & -857311 & -1597400 & -1411795 \\ \hline
        Asus Nexus 7 2013     & -857311 & -1624700 & -1411795 \\ \hline
        Samsung Galaxy Xcover & -879018 & -1631923 & -1420843 \\ \hline
        Sony Xperia active    & -894553 & -1633708 & --       \\ \hline
        Emulator              & -855010 & -1597400 & -1411448 \\ \hline
        Desktop computer      & -787082 & -1424148 & -1292057 \\ \hline
    \end{tabular}
    \centering
    \caption{The best soft score after 10 seconds time limit (larger is better).}
    \label{ScoreLimitTable}
\end{table}

Although, there is a significant difference between the desktop computer and Android phones or tablet, these mobile
devices are still sufficiently fast to use OptaPlanner tool and solve such kind of problem.

From testing and measuring, it can be said that portation of OptaPlanner is successful. Standard Java score calculation
of OptaPlanner is used and it is achieved satisfying times on the real mobile devices. Also, there were no problems
recorded with OptaPlanner tool and its use

\section{Future work}\label{FutureWorkSection}
This section describes future work which follows after this project. There are already plans how to improve OptaPlanner
integration to Android system and in addition, there are two main project which will be primarily processed.

\paragraph{OptaPlanner game}
Because OptaPlanner works on Android, there are many ways how to use it. One of them is to create a simple game where user
goal is to defeat the OptaPlanner by finding shortest way for vehicles. Using clicking on the customers user create road
for vehicle to the depot. Meanwhile time is measured and when user finish its work, OptaPlanner do the same and results
will be compared. This is the early proposal of the game which follows this work.

\paragraph{Drools} %TODO check the reference later
As noted in Chapter \ref{PortingChapter}, Drools package is not included in the poration due to its size and
complexity. There are plans to port Drools tool to Android separately. If the portation is successful, it will be
deployed and tested together with OptaPlanner. Although Drools tool is consuming a lot of computer resources, current
mobile devices already have high performance and it might be interesting to have such instrument on Android platform.
