tady popsat rozdil java jdk a android api
java beans problem
problemy android aplikaci


\section{Java SE API and Android API}
Android API je postaveno na Java SE 6, avšak jak je možno vidět na tabulce \todo{odkaz na tabulku} není kompletní Java SE API. Podstatná část balíčků chybí nebo jsou některé ne zcela kompletní. V případě balíčku uživatelského rozhraní jako je java.awt a java swing byly nahrazeny androidími grafickými uživatelskými prvky. Některé balíčky však nemusejí mít přímou souvislost s grafickým uživatelským rozhraním, a proto při portování aplikací můžeme narazit na problém s jejich nedispozicí.

\subsection{Java Beans}
Typick

\begin {table}[h!]
\begin{tabular}{|l|l|}
\hline
{\bf Java 6 SE Package} & {\bf Included in Android API} \\
\hline \hline
java.applet             & No -- missing completely    \\
java.awt                & Yes -- incomplete            \\
java.beans              & Yes -- incomplete            \\
java.io	                & Yes -- completed             \\
java.lang	            & Yes -- incomplete            \\
java.math	            & Yes -- completed             \\
java.net	            & Yes -- completed             \\
java.nio	            & Yes -- completed             \\
java.rmi	            & No  -- missing completely    \\
java.security	        & Yes -- incomplete            \\
java.sql	            & Yes -- completed             \\
java.text	            & Yes -- completed             \\
java.util	            & Yes -- incomplete            \\
javax.accessibility     & No -- missing completely    \\
javax.activation	    & No -- missing completely    \\
javax.activity	        & No -- missing completely    \\
javax.annotation	    & No -- missing completely    \\
javax.crypto	        & Yes -- complete              \\
javax.imageio	        & No -- missing completely    \\
javax.jws	            & No -- missing completely    \\
javax.lang	            & No -- missing completely    \\
javax.management        & No -- missing completely    \\
javax.naming	        & No -- missing completely    \\
javax.net	            & Yes -- complete              \\
javax.print		        & No -- missing completely    \\
javax.rmi		        & No -- missing completely    \\
javax.script	        & No -- missing completely    \\
javax.security          & Yes -- incomplete            \\
javax.sound             & No -- missing completely    \\
javax.sql	            & Yes -- incomplete javax.sql  \\
javax.swing	            & No -- missing completely    \\
javax.tools	            & No -- missing completely    \\
javax.transaction	    & No -- missing completely    \\
javax.xml	            & Yes -- incomplete            \\
org.ietf.jgss	        & No -- missing completely    \\
org.omg                 & No -- missing completely    \\
org.w3c.dom             & Yes -- incomplete            \\
org.xml.sax	            & Yes -- complete              \\
\hline
\end{tabular}
\centering
\caption{Java 6 SE packages in Android API}
\end{table}


\subsection{OpenBeans}
OpenBeans jsou redistribuce java.beans balíčku z Apache Harmony projektu. Namespace se ale od java.beans balíšku líší, v openbeans se používá com.googlecode.openbeans namespace. Vznikly v prvé řadě právě díky neexistenci java.beans na platformě Android. Jedná se o opensource projekt a je distribuován jako jar balíček, který je možné přidat do svého projektu.

\subsection{Mad Robot}
Podobný projekt jako je OpenBeans se nazývá Mad Robot. Stejně jako OpenBeans obsahuje redistribuci balíčku java.beans pod namespace com.madrobot.beans a navíc přidává i některé další balíčky např pro práci s databází, s grafikou, geometrií atd. Tento projekt je distribuován ve formě Maven závislostí.

\subsection{Jar Jar Links}
\todo{co je to jar jar}
\todo{jak ho nastavit}


\subsection{Core library flag}

\section{Návrh způsoby portace OptaPlanneru}
\subsection{}



