
\section{Testing}

\subsubsection{Devices}
The application was tested on several devices during and also after developement. Table \ref{TestingDevicesTable} shows
these testing devices and their important parameters. Some issues which are associated to them are described in
following paragraphs.

First one is Android version which specifies supported parts of Android features. Most of problems are related with
unsupported functions and they are ordinarily detected by compiler. However, from time to time some problems can appear
and therefore it is better to test application on various version of system.

Next parameters are display size and display resolution which together define density of screens points. Every device
has diferent display density and component layout must be well designed to fit on so many displays. Also after screen
rotation, components layout changes and it must be adapted.

Last parameters CPU and RAM defines device speed and computing capabilities. Because this application calculates with
many values, it was performed some tests and measurements which compare these devices.

\begin {table}[h!]
\begin{adjustwidth}{-1cm}{}
    \begin{tabular}{|l|c|c|c|c|}
        \hline
        \textbf{Device (Android version)} & \textbf{Display size} & \textbf{Display resolution} & \textbf{CPU} & \textbf{RAM} \\ \hline \hline
        LG Nexus 5 (5.1.0)            & 4.95 inches  & 1080 x 1920 pixels & 4 core 2.3 GHz & 2 GB   \\ \hline
        Asus Nexus 7 2013 (5.1.0)     & 7.0 inches   & 1200 x 1920 pixels & 4 core 1.5 GHz & 2 GB   \\ \hline
        Samsung Galaxy Xcover (4.1.2) & 4.0 inches   & 480 x 800 pixels   & 2 core 1 GHz   & 1 GB   \\ \hline
        Sony Xperia active (4.0.4)    & 3.0 inches   & 320 x 480 pixels   & 1 GHz          & 0.5 GB \\ \hline
    \end{tabular}
    \centering
    \caption{Testing devices}
    \label{TestingDevicesTable}
    \end{adjustwidth}
\end{table}

\paragraph{Time measurement}


\section{Future work}

\paragraph{OptaPlanner game}
Because OptaPlanner works on Android, following work which has already started is creating of simple game where user
goal is defeat the OptaPlanner by finding shortest way for vehicles. Using clicking on the customers user create road
for vehicle to the depot. Meanwhile time is measured and when user finish its work OptaPlanner do the same and results
will be compared. This is the early proposal of the game which follows this work.

\paragraph{Drools}
As was written in Chapter \ref{PortingChapter} Drools package was not included in poration due to its size and
complexity. There are plans to port Drools tool to Android separately and if portation is successful, it will be
deployed and tested together with OptaPlanner. Although the Drools tool is consuming lot of computer resources, current
mobile devices already have high performance and it might be interesting to have such instrument on Android platform.
