This chapter deals with porting of OptaPlanner to Android. Portation is a~modification of software for the purpose of
usage on different platforms. OptaPlanner is designed for Java Standard Edition (SE) platform and for integration to
Android it is necessary to compare both platforms and resolve potential problems.

In the first Section~\ref{RequirementsPortationSection}, requirements for OptaPlanner portation are introduced.
Comparison of Java Application Programming Interface (API) and Android API are described in
Section~\ref{ComparsionSection} and discovered JavaBeans problem is discussed in Section~\ref{JavaBeansProblemSection}.
Section~\ref{SummarySection} summarizes possible solutions of JavaBeans problem. Section~\ref{PortationDesignSection}
deals with design of the portation and the last Section~\ref{PortationImplSection} focuses on its implementation.

\section{Requirements for OptaPlanner portation}\label{RequirementsPortationSection}
In this section, three essential requirements for OptaPlanner portation are introduced. The first one focuses on
automatic build process, the second one discusses the use of Drools library and the last requirement targets on
OptaPlanner usability.

\paragraph{Automatic build process}
First of the portation requirements is ensuring of the automatic build process. When an~application is built, it is
good to make all the steps automatic. OptaPlanner libraries should be imported correctly and all the dependencies should
be included or should be prepared by build script. Any other problems of the portation has to also be resolved in a~way
which does not required much effort from users of the scripts. These procedures have to be described and demonstrated on
a~model application.

\paragraph{Drools library}
As described in Chapter~\ref{OptaPlannerChapter}, one of the ways of computing function in OptaPlanner is by the Drools
rules. Drools library is distributed as a~part of OptaPlanner and also as a~standalone project. Using this library, it
is possible to write rule prescriptions for score calculation. Another way how to compute the score is standard Java
calculation which does not require additional dependencies. Because Drools project is not optimized for mobile platforms
and there is another way how to calculate the score, it should not be included in the portation of OptaPlanner on
Android.

\paragraph{OptaPlanner usability}
In summary, it is necessary to prepare OptaPlanner for the Android platform to enable possibility of using tools for
solving planning problems. Mobile devices have limited computing capabilities and storage space in comparison to desktop
computers. Ported OptaPlanner should consider these capabilities and adapt to them.

\section{Java API packages comparison}\label{ComparsionSection}
This section compares two Java Application Programming Interfaces -- Java 6 Standard Edition API and Android API.
Depending on the comparison, OptaPlanner dependencies on Java SE API are identified on Android API and the consequent
problems are highlighted.

\paragraph{Java SE API and Android API}
Android API is based on Apache Harmony Java 6 SE API~\cite{Apache} (the open-source version of Java SE). First column of
Table~\ref{ApiDiffTable} shows all the packages in Java 6 SE API. In second column of the table, it is described whether
or not are the packages included in Android API. If both APIs are compared, it is certainly possible to say that Android
API is not complete Java SE API. Out of 38 Java SE API packages, 20 packages are completely missing and 9 packages are
incomplete in Android API. For example, packages for graphical user interface such as \texttt{java.swing} and
\texttt{java.awt} are not included and provided for Android development because they are replaced with the Android
graphical elements.

\paragraph{Android API and Optaplanner}
The third column in Table~\ref{ApiDiffTable} describes whether the package is needed by OptaPlanner. This tool directly
uses only 6 of the total number of 38 packages of Java SE API, namely: \texttt{java.beans}, \texttt{java.io},
\texttt{java.lang}, \texttt{java.math}, \texttt{java.net} and \texttt{java.util}. Three of the used packages are
incomplete in Android API but only one of them affects OptaPlanner, speciffically \texttt{java.beans} package. Due to
the fact that some of the \texttt{java.beans} classes are missing in Android API, direct OptaPlanner integration is
impossible. Other missing packages do not affect OptaPlanner use on Android.

\begin {table}[h!]
    \begin{tabular}{|l|l|c|}
        \hline
        \textbf{Java 6 SE Package} &
        \textbf{Included in Android API} &
        \textbf{Needed by OptaPlanner} \\ \hline \hline
        java.applet           & No -- missing completely  & No  \\ \hline
        java.awt              & Yes -- incomplete         & No  \\ \hline
        java.beans            & Yes -- incomplete         & Yes \\ \hline
        java.io               & Yes -- complete           & Yes \\ \hline
        java.lang             & Yes -- incomplete         & Yes \\ \hline
        java.math             & Yes -- complete           & Yes \\ \hline
        java.net              & Yes -- complete           & Yes \\ \hline
        java.nio              & Yes -- complete           & No  \\ \hline
        java.rmi              & No -- missing completely  & No  \\ \hline
        java.security         & Yes -- incomplete         & No  \\ \hline
        java.sql              & Yes -- complete           & No  \\ \hline
        java.text             & Yes -- complete           & No  \\ \hline
        java.util             & Yes -- incomplete         & Yes \\ \hline
        javax.accessibility   & No -- missing completely  & No  \\ \hline
        javax.activation      & No -- missing completely  & No  \\ \hline
        javax.activity        & No -- missing completely  & No  \\ \hline
        javax.annotation      & No -- missing completely  & No  \\ \hline
        javax.crypto          & Yes -- complete           & No  \\ \hline
        javax.imageio         & No -- missing completely  & No  \\ \hline
        javax.jws             & No -- missing completely  & No  \\ \hline
        javax.lang            & No -- missing completely  & No  \\ \hline
        javax.management      & No -- missing completely  & No  \\ \hline
        javax.naming          & No -- missing completely  & No  \\ \hline
        javax.net             & Yes -- complete           & No  \\ \hline
        javax.print           & No -- missing completely  & No  \\ \hline
        javax.rmi             & No -- missing completely  & No  \\ \hline
        javax.script          & No -- missing completely  & No  \\ \hline
        javax.security        & Yes -- incomplete         & No  \\ \hline
        javax.sound           & No -- missing completely  & No  \\ \hline
        javax.sql             & Yes -- incomplete         & No  \\ \hline
        javax.swing           & No -- missing completely  & No  \\ \hline
        javax.tools           & No -- missing completely  & No  \\ \hline
        javax.transaction     & No -- missing completely  & No  \\ \hline
        javax.xml             & Yes -- incomplete         & No  \\ \hline
        org.ietf.jgss         & No -- missing completely  & No  \\ \hline
        org.omg               & No -- missing completely  & No  \\ \hline
        org.w3c.dom           & Yes -- incomplete         & No  \\ \hline
        org.xml.sax           & Yes -- complete           & No  \\ \hline
    \end{tabular}
    \centering
    \caption{Java 6 SE API packages in Android API and OptaPlanner dependencies.}
    \label{ApiDiffTable}
\end{table}

\section{JavaBeans problem}\label{JavaBeansProblemSection}
JavaBeans package allows to reuse components written in the Java programming language. Primarily, its classes are used
for creation of graphical user interface but they can also be used for introspection of methods, properties and events.
More information about JavaBeans can be found in~\cite{Beans}.

OptaPlanner is completely written in the Java language and one of its dependencies is \texttt{java.beans} package. As
described in Section~\ref{ComparsionSection}, this package is incomplete and classes which OptaPlanner requires are
missing. In case of OptaPlanner use on Android, the compiler throws \texttt{ClassNotFoundException} and with this error,
OptaPlanner project cannot be built. In this section, the possible solutions of this problem are presented and one of
the solutions is selected for the needs of the portation.

The proposed solutions can be divided into two groups. First one contains solutions which do not interfere in
OptaPlanner source code and they are described in Section~\ref{RepackingJavaBeansSection},
Section~\ref{OpenJdkDistrSection} and Section~\ref{PrunedJarSection}. Solutions which change the source code of
OptaPlanner belong to the second group and they are described in Section~\ref{UseOpenBeansSection} and
Section~\ref{RemoveJavaBeansSection}.

\subsection{Repacking of JavaBeans redistribution to Java core namespace}\label{RepackingJavaBeansSection}
The first of the possibilities how to complete the missing \texttt{java.beans} package is use of a~JavaBeans
redistribution. These libraries are specially designed for the Android platform to support JavaBeans or other missing
packages. If OptaPlanner source code should stay the same, it is necessary to repackage these libraries to Java core
namespace. Java core namespace is an~identification of Java API classes which belongs to \texttt{java.*} or
\texttt{javax.*} namespace. In the following paragraphs, two redistribution and the Jar Jar Links tool for repacking
Java Archive (JAR) files are introduced. Last paragraph describes problem which is appearing during usage of classes
from Java core namespace on Android.

\paragraph{OpenBeans}
OpenBeans project~\cite{OpenBeans} is a~redistribution of \texttt{java.beans} package based on the Apache Harmony
project. It was created because of missing JavaBeans on the Android platform. Used namespace of this package is
\texttt{com.googlecode.openbeans}. OpenBeans is an~open-source project and it is distributed as JAR file that can be
included into a~Java or an~Android project.

\paragraph{Mad Robot}
A~similar project to OpenBeans is called Mad Robot~\cite{MadRobot}. As well as OpenBeans, it contains redistribution of
\texttt{java.beans} package in \texttt{com.madrobot.beans} namespace but it also includes some additional packages for
database, graphics or geometry manipulation. This project is distributed as Maven dependency.

\paragraph{Jar Jar Links}
Jar Jar Links~\cite{JarJar} is a~utility for repackaging Java libraries. It enables to repack Java classes from one
namespace to another. For proper use, it is necessary to define rules which describe way how Java classes should be
repacked and how the classes have to be placed in a~JAR file. Jar Jar tool uses this file to create new JAR with
repacked classes.

\paragraph{Core library problem}
Compilation of an~Android project that contains a~class from namespace \texttt{java.*} or \texttt{javax.*} crashes
during the translation which is highlighted by message about using the classes from Java core namespace. It is
a~protection against unauthorized use of the namespace. This can be avoided by using \texttt{--core-library} flag.
Adding the flag allows translation of the application. One of options how to use the flag is modifying the \texttt{dx}
script in Android SDK (Software Development Kit). Listing~\ref{FlagListing} shows the flag added to the last line of the
\texttt{dx} file.
\\
\begin{lstlisting}[captionpos={b},caption={Core library flag in the last line of \texttt{dx} script.},frame={lines},
label={FlagListing},basicstyle=\footnotesize]
exec java $javaOpts -jar "$jarpath" --core-library "$@"
\end{lstlisting}

\subsection{Use of OpenJDK distribution source code}\label{OpenJdkDistrSection}
This solution is based on the use of available source code of OpenJDK Java SE~\cite{OpenJDK} which is an~open-source
distribution of Java SE. By adding sources to an~Android project, it is possible to get the necessary libraries. The
advantage of this solution is that the dependency failures are seen in translation and not when the application runs.
This enables to choose only the required classes. However, this adjustment is not trivial. It needs to be done by
a~special tool that removes unused dependencies or it must be done manually.

\subsection{Use of pruned rt.jar from OpenJDK distribution}\label{PrunedJarSection}
The last option without interference to the source code is use of the \texttt{rt.jar} file which is part of the Java SE
libraries. This package contains JavaBeans compiled classes and other parts of Java SE. Due to its size, it is not well
suited for an~Android applications and it also includes libraries that are already contained in Android API and can
causes collisions. Therefore, it has to be pruned. The advantage of pruning is that there is no need to worry about
dependencies that are not needed for OptaPlanner tool because these files are not again compiled. On the other hand,
it may happen that an~application hits some missing required dependencies during runtime and the application crashes.

\subsection{Use of OpenBeans in OptaPlanner project}\label{UseOpenBeansSection}
This is the first solution which intervenes to the OptaPlanner source code and it consists of replacing all
\texttt{java.beans} dependencies for the \texttt{com.googlecode.openbeans} by rewriting all imports and by addition of
\texttt{OpenBeans.jar} archive to the OptaPlanner core project. This causes redirection of all dependencies to
OpenBeans. The disadvantage of this solution is the intervention in the OptaPlanner source code. In terms of Android
application developers, it is needed to create anew fork (or branch) of OptaPlanner and maintenance of such copy is
complicated and complex.

\subsection{Removing and replacing JavaBeans from OptaPlanner}\label{RemoveJavaBeansSection}
Last option to solve the JavaBeans problem is its elimination from the source code and its replacement by another
technology. This is the biggest intervention to OptaPlanner code of all the offered solutions and it is also the major
disadvantage. As in the previous solution, it is necessary to create a~new fork of OptaPlanner and take care of its
maintenance.

\section{Summary of approaches}\label{SummarySection}
Table~\ref{SummaryJavaBeansTable} shows summary of the JavaBeans problem solutions. The first column contains solution
name. In the second column, licenses which should be respected when using concrete approach are placed. The need for
modification of OptaPlanner source code is marked in the third column. Assumed solution level of difficulty is placed in
the fourth column and advantages and disadvantages of each approach are described in the last column.

License and avoiding the modification of OptaPlanner are the essential requirements of the selection. License has to
permit commercial and private use and modification of the code. Furthermore, the easier solution is better because then
each developer has an~opportunity to easily use OptaPlanner on Android without making any special modifications.

The best solution of JavaBeans problem seems to be repacking of the OpenBeans redistribution of JavaBean to Java core
namespace. In this approach, suitable Apache license must be respected. This license is free software license and it
allows easy usage of the code. It is not necessary to modify the OptaPlaner code and generally, this approach requires
less effort from the programmer.

\begin{table}[h!]
    \scalebox{0.95}{
        \begin{tabular}{|l|c|c|c|p{5cm}|}
            \hline
            \multicolumn{1}{|p{2.5cm}|}{\centering \textbf{Approach name}} &
            \textbf{License} &
            \multicolumn{1}{p{2.4cm}}{\centering \textbf{OptaPlanner modification}} &
            \multicolumn{1}{|p{1.8cm}}{\centering \textbf{Level of difficulty}} &
            \multicolumn{1}{|p{5cm}|}{\centering \textbf{Advantages and disadvantages}} \\ \hline \hline

            \multicolumn{1}{|p{2.5cm}|}{Repacking OpenBeans redistribution to Java core namespace} &
            \multicolumn{1}{p{2cm}|}{\centering Apache License 2.0} &
            No &
            Easy &
            \texttt{+} standalone jar file, no problems with dependencies \\ \hline

            \multicolumn{1}{|p{2.5cm}|}{Repacking of Mad Robot redistribution to Java core namespace} &
            LGPL 2.1 &
            No &
            Easy &
            \texttt{+} same as in previous case \\ \hline

            \multicolumn{1}{|p{2.5cm}|}{Use the OpenJDK distribution source code} &
            GPL 2.0 &
            No &
            Medium &
            \texttt{+} dependency failure occurs in translation, source code control

            \texttt{-} difficult adjustment which can cause problems with dependencies \\ \hline

            \multicolumn{1}{|p{2.5cm}|}{Use of pruned rt.jar from OpenJDK distribution} &
            GPL 2.0 &
            No &
            Hard &
            \texttt{+} standalone jar file

            \texttt{-} difficult adjustment which can cause problems with dependencies, incosistent jar, prunning
            \\ \hline

            \multicolumn{1}{|p{2.5cm}|}{Use of OpenBeans in OptaPlanner project} &
            \multicolumn{1}{p{2cm}|}{\centering Apache License 2.0} &
            Yes &
            Easy &
            \texttt{+} easy adjustment

            \texttt{-} need of modification of OptaPlanner source code and the subsequent maintenance of
            OptaPlanner fork \\ \hline

            \multicolumn{1}{|p{2.5cm}|}{Removing and replacing JavaBeans from OptaPlanner} &
            -- &
            Yes &
            Medium &
            \texttt{-} same as in previous case\\ \hline
        \end{tabular}
    }
    \centering
    \caption{Summary of the JavaBeans problem solutions.}
    \label{SummaryJavaBeansTable}
\end{table}

\section{Design of the portation}\label{PortationDesignSection}
This section describes design of the portation based on the requirements from
Section~\ref{RequirementsPortationSection}. Major task is to make the entire process automatic and show how OptaPlanner
can be ported into Android project of a~developer tool.

\paragraph{Developer tools}
As described in Chapter~\ref{AndroidChapter}, the official environment for development of Android applications is called
Android studio which brings new build system which is based on Gradle language. Android studio is selected as primary
developer tool for portation needs and Gradle laguage is chosen for automatization of all the required processes.

\paragraph{Importing of OptaPlanner libraries}
OptaPlanner is distributed as a~Maven dependency or it is packed in JAR files. Because one of the requirement is
automatic build process, first type of distribution is selected. Gradle language supports Maven dependencies and it
allows complete automation of importing process. Because OptaPlanner is daily developed, its last snapshot version
should be always used to secure that the portation is compatible with Android for the future releases of OptaPlanner.

\paragraph{Exclusion of unnecessary libraries}
As written in Section~\ref{RequirementsPortationSection}, Drools libraries should not be included in portation. It can
be assumed that also another libraries can be excluded because they can be already part of Android API. These exclusions
has to be performed by Gradle language to secure the process automation. It also helps to reduce size of the potential
applications.

\paragraph{Completion of missing libraries}
Section~\ref{JavaBeansProblemSection} describes JavaBeans problem and one of the suggested possible solutions was
selected in Section~\ref{SummarySection}. Task of the chosen solution is to repack OpenBeans redistribution to Java core
namespace and following steps should be performed by Gradle language to automate the entire process:

\begin{itemize}
    \item Download JarJar tool for repacking libraries.
    \item Download OpenBeans package which is in \texttt{com.googlecode.beans} namespace.
    \item Run JarJar tools and repack OpenBeans package to Java core namespace (\texttt{java.beans}).
    \item Clean temporary files and move final JAR into proper directory.
    \item Add \texttt{--core-library} flag to allow use of the Java core classes.
\end{itemize}

\section{Implementation of the portation}\label{PortationImplSection}
In this section, implementation of the portation is described. The first part focuses on the two main issues and the
second part on the Gradle scripts which implement the portation itself are introduced.

\subsection{JavaBeans and Core library flag}\label{JavaBeansCoreLibFlagSection}
Two main issues need be to solved before it is possible to use OptaPlanner on Android. First one is completion of
missing libraries to a~project and second one is adding the \texttt{--core-library} flag to allow use of the Java core
classes.

\paragraph{Completion of JavaBeans library}
Completion of JavaBeans library is implemented by Gradle scripts described in Section~\ref{GradleScriptsSection}. These
scripts can be used in any Android Gradle project and they enable to automatically download required tools and packages
from the Internet, repack OpenBeans library and create JavaBeans JAR file which is added to the project.

\paragraph{Core library flag}
If some classes from Java core namespace are present in an~Android project, its compilation fails. Proposed solution
where \texttt{--core-library} flag is added to \texttt{dx} script in Android SDK directory as shown in
Listing~\ref{FlagListing} does not work from version 1.1.0 of Android Gradle build tools. The build tools are changed
and they no longer use the \texttt{dx} script. Therefore, the flag must be directly embedded in settings of Android
plugin in \texttt{build.gradle} file as shown in Listing~\ref{CoreFlagGradleListing}.
\\
\begin{lstlisting}[captionpos={b},caption={Addition of Core library flag in \texttt{build.gradle} script.},frame={lines},
label={CoreFlagGradleListing},basicstyle=\footnotesize]
project.tasks.withType(com.android.build.gradle.tasks.Dex) {
    additionalParameters=['--core-library']
}
\end{lstlisting}

\subsection{Gradle scripts}\label{GradleScriptsSection}
The entire process of the portation is automatized by Gradle scripts. Gradle is default building tool for Android studio
which is currently a~primary application for creating Android projects.

This section discusses how OptaPlanner can be added to an~Android project and describes tasks which perform operations
required to complete missing libraries described in Section~\ref{PortationDesignSection}.  All the described tasks are
used in application described in Section~\ref{ApplicationChapter}.

\paragraph{OptaPlanner dependency}
As in any other project which does not use only standard Java classes, it is necessary to include third party libraries.
Gradle can use Maven dependencies and this is one of the ways how OptaPlanner is distributed. Libraries can also be
added directly as JAR files but it would be worse for automatization. As mentioned in
Section~\ref{RequirementsPortationSection}, Drools library should not be included and also \texttt{xmlpull} library has
to be excluded because it is already contained in Android API. Listing~\ref{DependencyListing} shows OptaPlanner
dependency with excluded libraries.
\\
\begin{lstlisting}[captionpos={b},caption={OptaPlanner Maven dependency in Gradle build script.},frame={lines},
label={DependencyListing},basicstyle=\footnotesize]
compile ('org.optaplanner:optaplanner-core:6.3.0-SNAPSHOT') {
    exclude group: 'xmlpull'
    exclude group: 'org.drools'
}
\end{lstlisting}

\paragraph{Downloading and removing tasks}
There are two downloading tasks \texttt{downloadJarJar} and \texttt{dowloadOpenBeans}. First task downloads repacking
tool Jar Jar links and second one downloads OpenBeans package in JAR format from the Internet. The tasks for removing
temporary files after their use are namely: \texttt{deleteJarJar}, \texttt{deleteOpenBeans}, \texttt{deleteRule} and
\texttt{deleteJavaBeans}.

\paragraph{Task for creating of repacking rule file}
Jar Jar tool needs to know how to repack \texttt{OpenBeans.jar}. For this purpose, there is \texttt{createRuleFile} task
which creates file with rule as shown in Listing~\ref{RuleListing}. This rule tells that all OpenBeans namespaces should
be replaced with JavaBeans namespace.
\\
\begin{lstlisting}[captionpos={b},caption={Jar Jar Links rule for repacking OpenBeans to Java core namespace.},
frame={lines},label={RuleListing},basicstyle=\footnotesize]
rule com.googlecode.openbeans.** java.beans.@1
\end{lstlisting}

\paragraph{Task for repacking OpenBeans}
Last task for manipulation with OpenBeans package is \texttt{repackOpenBeans} task. This task launches all the previous
tasks and also executes command which creates JAR file with repacked OpenBeans to Java core name space. The new file is
placed into project \texttt{/lib} directory. This command is shown in Listings~\ref{CommandListing}. Finally,
removing tasks are activated and temporary files are deleted.
\\
\begin{lstlisting}[captionpos={b},caption={Command for repacking \texttt{openbeans.jar} file.},frame={lines},
label={CommandListing},basicstyle=\footnotesize]
java -jar jarjar.jar process rule.txt openbeans.jar javabeans.jar
\end{lstlisting}

\paragraph{Task for adding Core library flag}
This task is originally created for modification of \texttt{dx} script in build tools directory in Android SDK. It finds
\texttt{dx} file and adds \texttt{--core-library} flag to the last line of the file. In latest versions of Android
Gradle build plugin, this approach does not work and it is replaced as described above in this section.
