This chapter deals with differences between Java SE API and Android API. Although Java libraries of Andoid API are based on Java SE, a lot of libraries are missing or they are incomplete. It causes problems that are dealt with in this chapter. These libraries must be added to Android project or dependencies to libraries have to be removed from the OptaPlanner core.

Portation is modification of software for the purpose of usage on different platforms. Optaplanner is designed for Java SE platform and for integration on Android platform it is necessary to check the API to find what difference are and solve the problems that occurs.

In the first Section \ref{comparsion}, Java SE API and Android API are compared. There are also shown packages which are used by OptaPlanner and JavaBeans problem is introduces. The proposals for solution of JavaBean problem are described in Section \ref{JavaBeans} and its summary is presented in the last Section~\ref{summary}.

\section{Requirements for OptaPlanner portation}
In this section, three essential requirements for OptaPlanner portation are introduced. The first one focuses on
automatic build process, the second one discusses the use of Drools library and the last requirement targets on
OptaPlanner usability.

\paragraph{Automatic build process}
First of the requirements of portation is ensuring of the automatic build process. When an application is built, it is
good to make all automatic. OptaPlanner libaries should be correctly imported and all the dependencies should be
included or should be prepared by build script. Any other problems of the portation has to be also resolved in a way
which does not required much effort from side of programmer. These procedures have to be described and demonstrated
on a sample application.

\paragraph{Drools library}
As is described in Chapter \ref{OptaPlannerChapter}, one of the ways of calculation function in OptaPlanner is by the
Drools rules. Drools library is distributed as a component of OptaPlanner and also as a stand-alone project. Using this
library, it is possible to write rule prescriptions for score calculation. Ahother way how to compute the score is
standart Java calculation which does not require additional dependencies. Because Drools project is not optimized for
mobile platforms and there is another way how to calculate the score, it should not be included in the portation of
OptaPlanner to Android.

\paragraph{Optaplanner usability}
In summary, it is necessary to prepare OptaPlanner for the Android platform to enable possibility of using tools for
solving planning problems. Mobile devices have limited computing capabilities and storage space in comparison to desktop
computers. Ported OptaPlanner should consider these capabilities and adapt to them.

\section{Java API packages comparsion}\label{ComparsionSection}
This section compares two Java Application Programming Interfaces (API) -- Java 6 Standart Edition API and Android API.
Depending on the comparison, OptaPlanner dependencies on Java SE API are identified on Android API and the consequent
problems are highlighted.

\paragraph{Java SE API and Android API}
Android API is based on Apache Harmony Java 6 SE API~\cite{Apache} (the open-source version of Java SE). First column of
Table~\ref{ApiDiffTable} shows all the packages in Java 6 SE API. In second column of the table, it is described whether
the packages are or are not included in Android API. If both APIs are compared, it is certainly possible to say that
Android API is not complete Java SE API. Out of 38 Java SE API packages, 20 packages are completely missing and 9
packages are incomplete in Android API. For example, packages for graphical user interface such as \texttt{java.swing}
and \texttt{java.awt} are not included and provided for Android development because they are replaced with the Android
graphical elements.

\paragraph{Android API and Optaplanner}
Third column in Table~\ref{ApiDiffTable} describes if specific package is needed by OptaPlanner. This tool directly uses
only 6 of the total number of 38 packages of Java SE API, namely: \texttt{java.beans}, \texttt{java.io},
\texttt{java.lang}, \texttt{java.math}, \texttt{java.net} and \texttt{java.util}. Three of the used packages are
incomplete in Android API but only one of them affects OptaPlanner and that is \texttt{java.beans} package. Due to the
fact that some of the \texttt{java.beans} classes are missing in Android API, direct OptaPlanner integration is
impossible. Other missing packages do not affect OptaPlanner use on Android.

\begin {table}[h!]
    \begin{tabular}{|l|l|c|}
        \hline
        \textbf{Java 6 SE Package} &
        \textbf{Included in Android API} &
        \textbf{Needed by OptaPlanner} \\ \hline \hline
        java.applet           & No -- missing completely  & No  \\ \hline
        java.awt              & Yes -- incomplete         & No  \\ \hline
        java.beans            & Yes -- incomplete         & Yes \\ \hline
        java.io               & Yes -- complete           & Yes \\ \hline
        java.lang             & Yes -- incomplete         & Yes \\ \hline
        java.math             & Yes -- complete           & Yes \\ \hline
        java.net              & Yes -- complete           & Yes \\ \hline
        java.nio              & Yes -- complete           & No  \\ \hline
        java.rmi              & No -- missing completely  & No  \\ \hline
        java.security         & Yes -- incomplete         & No  \\ \hline
        java.sql              & Yes -- complete           & No  \\ \hline
        java.text             & Yes -- complete           & No  \\ \hline
        java.util             & Yes -- incomplete         & Yes \\ \hline
        javax.accessibility   & No -- missing completely  & No  \\ \hline
        javax.activation      & No -- missing completely  & No  \\ \hline
        javax.activity        & No -- missing completely  & No  \\ \hline
        javax.annotation      & No -- missing completely  & No  \\ \hline
        javax.crypto          & Yes -- complete           & No  \\ \hline
        javax.imageio         & No -- missing completely  & No  \\ \hline
        javax.jws             & No -- missing completely  & No  \\ \hline
        javax.lang            & No -- missing completely  & No  \\ \hline
        javax.management      & No -- missing completely  & No  \\ \hline
        javax.naming          & No -- missing completely  & No  \\ \hline
        javax.net             & Yes -- complete           & No  \\ \hline
        javax.print           & No -- missing completely  & No  \\ \hline
        javax.rmi             & No -- missing completely  & No  \\ \hline
        javax.script          & No -- missing completely  & No  \\ \hline
        javax.security        & Yes -- incomplete         & No  \\ \hline
        javax.sound           & No -- missing completely  & No  \\ \hline
        javax.sql             & Yes -- incomplete         & No  \\ \hline
        javax.swing           & No -- missing completely  & No  \\ \hline
        javax.tools           & No -- missing completely  & No  \\ \hline
        javax.transaction     & No -- missing completely  & No  \\ \hline
        javax.xml             & Yes -- incomplete         & No  \\ \hline
        org.ietf.jgss         & No -- missing completely  & No  \\ \hline
        org.omg               & No -- missing completely  & No  \\ \hline
        org.w3c.dom           & Yes -- incomplete         & No  \\ \hline
        org.xml.sax           & Yes -- complete           & No  \\ \hline
    \end{tabular}
    \centering
    \caption{Java 6 SE API packages in Android API and Optaplanner dependencies on these packages.}
    \label{ApiDiffTable}
\end{table}

\section{JavaBeans problem}\label{JavaBeans}
JavaBeans package allows to reuse components written in the Java programming language. Mostly, its classes are used for
creation of graphical user interface but they can also be used for introspection of methods, properties and events.
More information about JavaBeans can be found in~\cite{Beans}.

OptaPlanner is completely written in the Java language and one of its dependencies is \texttt{java.beans} package. As
described in Section~\ref{ComparsionSection}, this package is incomplete and classes which OptaPlanner requires are
missing. In case of OptaPlanner use on Android, the compiler throws \texttt{ClassNotFoundException} and with this error,
OptaPlanner project cannot be built. In this section, the possible solutions of this problem are presented and one of
the solutions is selected for needs of the portation.

The proposed solutions can be divided into two groups. First one contains solutions which do not interfere into
OptaPlanner source code and they are described in Section~\ref{RepackingJavaBeansSection},
Section~\ref{OpenJdkDistrSection} and Section~\ref{PrunedJarSection}. Solution which change the source code of
OptaPlanner belongs to the second group and they are described in Section~\ref{UseOpenBeansSection} and
Section~\ref{RemoveJavaBeansSection}.

\subsection{Repacking of JavaBeans redistribution to Java core namespace}\label{RepackingJavaBeansSection}
The first of the ways how to complete the missing \texttt{java.beans} package is use of a JavaBeans redistribution.
These libraries are specially designed for the Android platform to support JavaBeans or other missing packages. If
OptaPlanner source code should stay the same, it is necessary to repackage these libraries to Java core namespace. Java
core namespace is an identification of Java API classes which belongs to \texttt{java.*} or \texttt{javax.*} namespace.
In the following paragraphs, two redistribution and the Jar Jar Links tool for repacking Java Archive (JAR) files are
introduced. Last paragraph describes problem which appears when classes from Java core namespace are used on Android.

\paragraph{OpenBeans}
OpenBeans project \cite{OpenBeans} is a redistribution of \texttt{java.beans} package based on the Apache Harmony
project. It was created because of missing JavaBeans on the Android platform. Used namespace of this redistribution is
\texttt{com.googlecode.openbeans}. OpenBeans is an open-source project and it is distributed as JAR file that can be
included into a Java or an Android project.

\paragraph{Mad Robot}
A similar project to OpenBeans is called Mad Robot \cite{MadRobot}. As well as OpenBeans, it contains redistribution of
\texttt{java.beans} package in \texttt{com.madrobot.beans} namespace but it also includes some additional packeges for
database, graphics or geometry manipulation. This project is distributed as Maven dependency.

\paragraph{Jar Jar Links}
Jar Jar Links \cite{JarJar} is a utility for repackaging Java libraries. It enables to repack Java classes from one
namespace to another. For proper use, it is necessary to define rules which describe way how Java classes should be
repacked and the classes have to be placed in a JAR file. Jar Jar tool uses this file to create new JAR with repacked
classes.

\paragraph{Core library problem}
Compilation of Android project that contains a class from namespace \texttt{java.*} or \texttt{javax.*} crashes during
the translation which is highlighted by message about using of a classes from Java core namespace. It is a protection
against unauthorized use of the namespace. This can be avoided by using \texttt{--core-library} flag. Adding the flag
allows translation of an application.
%\begin{lstlisting}[captionpos={b},caption={Spanning tree broadcast algorithm.},frame={lines},label={lastline},basicstyle=\footnotesize]
%exec java $javaOpts -jar "$jarpath" --core-library "$@"
%\end{lstlisting}

\subsection{Use of OpenJDK distribution source code}\label{OpenJdkDistrSection}
This solution is based on the use of available source code of OpenJDK Java SE \cite{OpenJDK} which is an open-source
distribution of Java SE. By adding sources to an Android project, it is possible to get the necessary libraries. The
advantage of this solution is that the dependency failure are seen in translation and not when the application runs.
This makes possible to choose only the required classes. However, this adjustment is not trivial. It needs to be done by
a special tool that removes unused dependencies or it must be done manually.

\subsection{Use of pruned rt.jar from OpenJDK distribution}\label{PrunedJarSection}
The last option without interference to the source code is use of the rt.jar file which is part of the Java SE
libraries. This package contains JavaBeans compiled classes and other parts of Java SE. Due to its size, it is not well
suited for an Android applications and it also includes libraries that are already contained in Android API and it can
causes collisions. Therefore, it has to be pruned. The advantage of pruning is that there is no need to worry about
dependencies that are not needed for Optaplanner tool because these files are not again compiled. On the other hand,
it may happen that an application hits some missing required dependencies during runtime and the application crashes.

\subsection{Use of OpenBeans in OptaPlanner project}\label{UseOpenBeansSection}
This is the first solution which intervenes to the Optaplanner source code and it consists of replacing all
\texttt{java.beans} dependencies for the \texttt{com.googlecode.openbeans} by rewriting all imports and by addition of
OpenBeans.jar archive to the Optaplanner core project. This causes that all dependencies are redirected to OpenBeans.
The disadvantage of this solution is the intervention to OptaPlanner source code. In terms of Android application
developers, it is needed to create a new fork of OptaPlanner and modify it and this causes that the maintenance is then
considerably complicated.

\subsection{Removing and replacing JavaBeans from OptaPlanner}\label{RemoveJavaBeansSection}
Last option to solve the JavaBeans problem is its elimination from the source code and its replacing by another
technology. This is the biggest intervention to Optaplanner code of the offered solutions and it is also the major
disadvantage. As in the previous solution, it is necessary to create a new fork of OptaPlanner and take care of its
maintenance.

\section{Summary of approaches}\label{SummarySection}
Table \ref{SummaryJavaBeansTable} shows summary of the JavaBeans problem solutions. The first column contains solution
name. In the second column, licenses which should be respected when using concrete approach are placed. The need for
modification of OptaPlanner source code is marked in the third column. Assumed solution level of difficulty is place in
the fourth column and advantages and disadvantages of each approach are described in the last column.

The essential requirements of the selection are license and avoiding the modification of OptaPlanner. License have to
permits commercial and private use and modification of the code. Furthermore, the easier solution is better solution
because then each developer has an opportunity to easily use OptaPlanner on Android without making any special
modifications.

The best solution of JavaBeans problem seems to be repacking of the OpenBeans redistribution of JavaBean to Java core
namespace. In this approach, suitable Apache licence must be respected. This licence is free software license and it
allows easily use the code. It is not necessary to modify the OptaPlaner code and generally, this approach requires less
effort from the programmer.

\begin {table}[h!]
    \scalebox{0.95}{
        \begin{tabular}{|l|c|c|c|p{5cm}|}
            \hline
            \multicolumn{1}{|p{2.5cm}|}{\centering \textbf{Approach name}} &
            \textbf{Licence} &
            \multicolumn{1}{p{2.4cm}}{\centering \textbf{Optaplanner modification}} &
            \multicolumn{1}{|p{1.8cm}}{\centering \textbf{Level of difficulty}} &
            \multicolumn{1}{|p{5cm}|}{\centering \textbf{Advantages and disadvantages}} \\ \hline \hline

            \multicolumn{1}{|p{2.5cm}|}{Repacking OpenBeans redistribution to Java core namespace} &
            \multicolumn{1}{p{2cm}|}{\centering Apache License 2.0} &
            No &
            Easy &
            \texttt{+} standalone jar file, no problems with dependencies \\ \hline

            \multicolumn{1}{|p{2.5cm}|}{Repacking of Mad Robot redistribution to Java core namespace} &
            LGPL 2.1 &
            No &
            Easy &
            \texttt{+} same as in previous case \\ \hline

            \multicolumn{1}{|p{2.5cm}|}{Use the OpenJDK distribution source code} &
            GPL 2.0 &
            No &
            Medium &
            \texttt{+} dependency failure occurs in translation, source code control

            \texttt{-} difficult adjustment which can cause problems with dependencies \\ \hline

            \multicolumn{1}{|p{2.5cm}|}{Use of pruned rt.jar from OpenJDK distribution} &
            GPL 2.0 &
            No &
            Hard &
            \texttt{+} standalone jar file

            \texttt{-} difficult adjustment which can cause problems with dependencies, incosistent jar, prunning
            \\ \hline

            \multicolumn{1}{|p{2.5cm}|}{Use of OpenBeans in OptaPlanner project} &
            \multicolumn{1}{p{2cm}|}{\centering Apache License 2.0} &
            Yes &
            Easy &
            \texttt{+} easy adjustment

            \texttt{-} need of modification of Optaplanner source code and the subsequent maintenance of
            OptaPlanner fork \\ \hline

            \multicolumn{1}{|p{2.5cm}|}{Removing and replacing JavaBeans from OptaPlanner} &
            -- &
            Yes &
            Medium &
            \texttt{-} same as in previous case\\ \hline
        \end{tabular}
    }
    \centering
    \caption{Summary of the JavaBeans problem solutions.}
    \label{SummaryJavaBeansTable}
\end{table}
