
\section{Testing}

\subsubsection{Devices}
The application was tested on several devices during and also after developement. Table \ref{TestingDevicesTable} shows
these testing devices and their important parameters. Some issues which are associated to them are described in
following paragraphs.

First one is Android version which specifies supported parts of Android features. Most of problems are related with
unsupported functions and they are ordinarily detected by compiler. However, from time to time some problems can appear
and therefore it is better to test application on various version of system.

Next parameters are display size and display resolution which together define density of screens points. Every device
has diferent display density and component layout must be well designed to fit on so many displays. Also after screen
rotation, components layout changes and it must be adapted.

Last parameters CPU and RAM defines device speed and computing capabilities. Because this application calculates with
many values, it was performed some tests and measurements which compare these devices.

\begin {table}[h!]
\begin{adjustwidth}{-1cm}{}
    \begin{tabular}{|l|c|c|c|c|}
        \hline
        \textbf{Device (Android version)} & \textbf{Display size} & \textbf{Display resolution} & \textbf{CPU} & \textbf{RAM} \\ \hline \hline
        LG Nexus 5 (5.1.0)            & 4.95 inches  & 1080 x 1920 pixels & 4 core 2.3 GHz & 2 GB   \\ \hline
        Asus Nexus 7 2013 (5.1.0)     & 7.0 inches   & 1200 x 1920 pixels & 4 core 1.5 GHz & 2 GB   \\ \hline
        Samsung Galaxy Xcover (4.1.2) & 4.0 inches   & 480 x 800 pixels   & 2 core 1 GHz   & 1 GB   \\ \hline
        Sony Xperia active (4.0.4)    & 3.0 inches   & 320 x 480 pixels   & 1 GHz          & 0.5 GB \\ \hline
    \end{tabular}
    \centering
    \caption{Testing devices}
    \label{TestingDevicesTable}
    \end{adjustwidth}
\end{table}

\subsubsection{Testing measurement}
After debugging the application, two measurements were performed on devices mentioned in Table \ref{TestingDevicesTable}
and one desktop computer. Parameters of the computer was:

\begin{itemize}
\item \textbf{Device} -- Notebook Lenovo ThinkPad T430s
\item \textbf{Processor} -- Intel Core i7-3520M 2.90GHz
\item \textbf{RAM} -- 16 GB
\item \textbf{OS} -- Fedora 21 64b
\end{itemize}

For both tests, three testing vrp file was used. They differ with number of customers and number and capacity of
vehicles. Parameters of these files are shown in table \ref{TestingFilesTable}. All the cases was measured five times
and it was created diameter of values in case of first measurement or best reached score was used in case of second
measurement.

\begin {table}[h!]
    \begin{tabular}{|l|c|c|c|}
        \hline
        \textbf{Vrp file} & \textbf{Number of costumers} & \textbf{Number of vehicles} & \textbf{Vehicle capacity} \\ \hline \hline
        A-n32-k5.vrp      & 32   & 5 & 100   \\ \hline
        A-n64-k9.vrp      & 64   & 9 & 100   \\ \hline
        F-n135-k7.vrp     & 135  & 7 & 2210  \\ \hline
    \end{tabular}
    \centering
    \caption{Testing files}
    \label{TestingFilesTable}
\end{table}

First test dealt with time of first found solution. Every test was measure five times and it was created diameter which
is written in Table \ref{FirstFoundTable}. As can be seen, difference between desktop computer and mobile devices is
quite substantial and with more complicated problem difference between these devices increasing. It can also be observed
that more powerful mobile devices surpass older device with worse CPU.

\begin {table}[h!]
    \begin{tabular}{|l|c|c|c|}
        \hline
        \textbf{Device}       & \textbf{A-n32-k5.vrp} & \textbf{A-n64-k9.vrp} & \textbf{F-n135-k7.vrp} \\ \hline \hline
        LG Nexus 5            & 0,162 s               & 0,561 s               & 2,760 s                \\ \hline
        Asus Nexus 7 2013     & 0,228 s               & 0,795 s               & 4,132 s                \\ \hline
        Samsung Galaxy Xcover & 0,411 s               & 1,434 s               & 8,219 s                \\ \hline
        Sony Xperia active    & 0,755 s               & 2,705 s               & 16,334 s               \\ \hline
        Desktop computer      & 0,075 s               & 0,130 s               & 0,284 s                \\ \hline
    \end{tabular}
    \centering
    \caption{Time of first found solution}
    \label{FirstFoundTable}
\end{table}

Second test is displayed in Table \ref{ScoreLimitTable} and it was based on 10s time limit. After the time limit the
best soft score was saved and from five measurements the lowest was selected. Soft score shows distace which should be
traveled with all of the vehicles. Quite substantial difference is appeared again between mobile devices and desktop
computer. Although nexus devices have different parameters they reached same results in two cases. The oldest device
Sony Xperia active did not reach any result in that time limit for the last case.

\begin {table}[h!]
    \begin{tabular}{|l|c|c|c|}
        \hline
        \textbf{Device}       & \textbf{A-n32-k5.vrp} & \textbf{A-n64-k9.vrp} & \textbf{F-n135-k7.vrp} \\ \hline \hline
        LG Nexus 5            & 857311                & 1597400               & 1411795                \\ \hline
        Asus Nexus 7 2013     & 857311                & 1624700               & 1411795                \\ \hline
        Samsung Galaxy Xcover & 879018                & 1631923               & 1420843                \\ \hline
        Sony Xperia active    & 894553                & 1633708               & --                     \\ \hline
        Desktop computer      & 787082                & 1424148               & 1292057                \\ \hline
    \end{tabular}
    \centering
    \caption{Soft score after 10s limit}
    \label{ScoreLimitTable}
\end{table}

Although, there is significant difference between desktop computer and Android phones or tablet, these mobile devices
still sufficient fast to use OptaPlanner tool and solve such kind of problem.




\section{Future work}

\paragraph{OptaPlanner game}
Because OptaPlanner works on Android, following work which has already started is creating of simple game where user
goal is defeat the OptaPlanner by finding shortest way for vehicles. Using clicking on the customers user create road
for vehicle to the depot. Meanwhile time is measured and when user finish its work OptaPlanner do the same and results
will be compared. This is the early proposal of the game which follows this work.

\paragraph{Drools}
As was written in Chapter \ref{PortingChapter} Drools package was not included in poration due to its size and
complexity. There are plans to port Drools tool to Android separately and if portation is successful, it will be
deployed and tested together with OptaPlanner. Although the Drools tool is consuming lot of computer resources, current
mobile devices already have high performance and it might be interesting to have such instrument on Android platform.
